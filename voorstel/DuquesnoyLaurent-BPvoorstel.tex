%==============================================================================
% Sjabloon onderzoeksvoorstel bachproef
%==============================================================================
% Gebaseerd op document class `hogent-article'
% zie <https://github.com/HoGentTIN/latex-hogent-article>

% Voor een voorstel in het Engels: voeg de documentclass-optie [english] toe.
% Let op: kan enkel na toestemming van de bachelorproefcoördinator!
\documentclass{hogent-article}

% Invoegen bibliografiebestand
\addbibresource{voorstel.bib}

% Informatie over de opleiding, het vak en soort opdracht
\studyprogramme{Professionele bachelor toegepaste informatica}
\course{Bachelorproef}
\assignmenttype{Onderzoeksvoorstel}
% Voor een voorstel in het Engels, haal de volgende 3 regels uit commentaar
% \studyprogramme{Bachelor of applied information technology}
% \course{Bachelor thesis}
% \assignmenttype{Research proposal}

\academicyear{2023-2024}

\title{Optimizing administrative workflow in legal offices: a unified web interface approach}

\author{Laurent Duquesnoy}
\email{laurent.duquesnoy@student.hogent.be}

% TODO: Geef de co-promotor op
\supervisor[Co-promotor]{S. Dewolf (Deltalex, \href{mailto:stijn@deltalex.be}{stijn@deltalex.be})}

% Binnen welke specialisatierichting uit 3TI situeert dit onderzoek zich?
% Kies uit deze lijst:
%
% - Mobile \& Enterprise development
% - AI \& Data Engineering
% - Functional \& Business Analysis
% - System \& Network Administrator
% - Mainframe Expert
% - Als het onderzoek niet past binnen een van deze domeinen specifieer je deze
%   zelf
%
\specialisation{Mobile \& Enterprise development}
\keywords{Lawyer, Legal, Automation}

\begin{document}

\begin{abstract}
  Dit onderzoek richt zich op het verminderen van de administratieve werklast binnen advocatenkantoren door geautomatiseerde processen en integratie in de huidige software. 

  De focus, allereerst, ligt op het onderzoeken van huidige technologieën en tools die advocaten gebruiken voor hun administratieve taken. Vervolgens worden geautomatiseerde toepassingen ontwikkeld om specifieke bottlenecks weg te werken. 

  De studie omvat een testfase waarin de effectiviteit van deze oplossingen wordt geëvalueerd. Dit gebeurt door feedback te verzamelen van een of meerdere vennoten van het advocatenkantoor. De bedoeling is om de interface en de onderliggende componenten verder te optimaliseren voor een optimale workflow. 

  De voorgaande technische analyse zal een unieke kans zijn voor de onderzoeker om hun skills bij te scherpen op zowel het informaticadomein als in de administratieve/juridische sector. De proef zal 12 weken in beslag nemen en is opgedeeld in requirementanalyse, ontwerp en technologieselectie, implementatie en een uiteindelijke evaluatie waar de voor- en nadelen van de proof of concept worden besproken. 
\end{abstract}
\tableofcontents

% De hoofdtekst van het voorstel zit in een apart bestand, zodat het makkelijk
% kan opgenomen worden in de bijlagen van de bachelorproef zelf.
%---------- Inleiding ---------------------------------------------------------
\section{Introductie}%
\label{sec:introductie}

Mijn bachelorproef spitst zich toe op het optimaliseren van de administratieve workload bij advocatenkantoor Deltalex. Dit zal gebeuren via het invoeren van geautomatiseerde processen, het
ontwikkelen van een centrale webinterface om opzoekwerk te verrichten, daarbij ook een portaal om de sjablonen van de meestgebruikte documenten bij te houden en te bewerken.

Dit onderzoek vindt zijn oorsprong in de vaststelling van veel tijdrovende en repetitieve taken die de efficiëntie van dit kantoor kunnen belemmeren,
zoals het handmatig opzoeken van cliëntgegevens in openbare databases en het bijwerken van veelgebruikte documenttemplates.

De onderzoeksvraag van mijn bachelorproef luidt: "Hoe kunnen geautomatiseerde processen, in combinatie met een webinterface voor beheer, de administratieve
last in advocatenkantoren verminderen en de efficiëntie bevorderen?"

Het beoogde eindresultaat omvat niet alleen een werkend prototype van automatisatie en de bijbehorende webinterface,
maar ook een grondig rapport met aanbevelingen voor implementatie in advocatenkantoren.

Het succes van deze bachelorproef zal worden beoordeeld aan de hand van daadwerkelijke verbeteringen in efficiëntie en
tijdsbesparing binnen advocatenkantoor Deltalex.

%---------- Stand van zaken ---------------------------------------------------
\section{Literatuurstudie: automatisatie en de selectie van CMS's in de advocatuur}%
\label{sec:state-of-the-art}

Consider one real-world example: if your firm had 12,000 documents to review, and each one had 75 questions to answer, the task will take weeks to complete. Even with a sizable team working in concert, the sheer volume of work is daunting. One of those, “Where do I even start?” types of tasks others might shrink from.
But what if you started by working the process, not just digging into the work? By automating even a part of a task that large, you stand to shave hours off the time involved.
\autocite{ThomsonReuters2023}

Deze citatie illustreert dat investeren in een efficiëntere aanpak van je repetitieve taken zeker kan lonen. Dit is natuurlijk ook toepasbaar op een advocatenkantoor, volgende citatie illustreert het maken van documenten, waaronder dagvaardingen, invorderingen en mails.

Document creation, for example, requires a large amount of repetitive work. Diving right into the work may seem like the best way to get ahead, but by taking the time to define the fields in a document, and set up corresponding variables, it’s easy for a computer to automatically create custom, accurate documents on a scale that no human can match.
\autocite{ThomsonReuters2023}

Automatisatie kan ook executie van taken zoals indienprocedures en legal research exponentieel sneller maken. \autocite{Aston2023}

Het huidige landschap van administratieve\\ tools is heel groot. In het geval van Deltalex (en andere kantoren) hebben ze gekozen voor een bepaald softwarepakket. Dit is veelal een moeilijke keuze omdat er heel veel pakketten beschikbaar zijn en deze bieden elk hun voor- en nadelen. De selectie van een dergelijke tool kan een moeizaam proces zijn, maar kan gegoten worden in volgende stappen.
\autocite{Clio2023}

\begin{itemize}
	\item \emph{Kies tussen cloud- of lokale opslag:} Waar hosten? 
        \item \emph{Bekijk de kosten- en onderhoudsverschillen:} Varieert sterk op basis van stap 1.
        \item \emph{Zorg voor toegang tot informatie op afstand:} Veel advocaten werken soms op afstand, bv. vanuit een rechtbank.
        \item \emph{Controleer de compatibiliteit met bestaande software:}\\ Complexere pakketten worden meestal pas aangekocht als het kantoor zich verder bevindt dan de startfase. M.a.w. er is zeker al bestaande data op oudere systemen.
        \item \emph{Evalueer gebruiksgemak en trainingsvereisten:} Het is belangrijk dat iedere gebruiker zo snel en probleemloos met de nieuwe software aan de slag kan.
        \item \emph{Zorg voor ethische naleving en beveiliging:} Advocatenkantoren werken onder strikte regels voor professioneel gedrag, daarom is het aangeraden om voor gespecialiseerde software te opteren.
\end{itemize}

Dergelijke pocessen kunnen lang duren en in het geval van Deltalex is dit al gebeurd. Daarom bespreek ik in deze bachelorproef de ontwikkeling van een externe interface om specifieke taken sneller af te handelen.

%---------- Methodologie ------------------------------------------------------
\section{Methodologie}%
\label{sec:methodologie}
Deze bachelorproef zal een combinatie zijn van een case study in advocatenkantoor Deltalex en een proof of concept van het platform om in het kantoor de administratieve workload zoveel mogelijk te minimaliseren. Deze implementatie valideert de haalbaarheid en de effectiviteit van de oplossing doorheen heel het kantoor, beginnende bij één specifieke advocaat. 

Het onderzoek zal zich concentreren op accurate en objectieve resultaten en vermijdt onderzoekstechnieken die subjectieve data yielden, zoals enquêtes. Literatuurstudie, interviews en praktische data zullen de bouwstenen zijn van het uiteindelijke verdict van de implementatie. 

De voorgaande technische analyse zal voor mij een unieke kans zijn om op zowel het informaticadomein als in de administratieve/ juridische sector mijn skills bij te scherpen. Er zullen tools gebruikt worden als front-end-webassembly talen voor een moderne approach van een webinterface. Voor automatisatie van queries online zal er gebruik gemaakt worden van scripttalen en er zal ook grondig uitgezocht worden hoe men sjablonen zo effectief mogelijk kan aanpassen via MS Office XML-injecties in documenten. 

Deze proef zal 12 weken in beslag nemen en is optimaal zoals onderstaand ingedeeld:
\begin{itemize}
        \item Requirementanalyse: 2 weken.\\ \emph{Deliverable} -> Verslag van requirements
        \item Ontwerp en technologieselectie: 2 weken.\\ \emph{Deliverable} -> technologiestack en praktische planning voor implementatie
        \item Implementatie: 4 weken.\\ \emph{Deliverable} -> werkende proof of concept
        \item Evaluatie: 2 weken. \\ \emph{Deliverable} -> Evaluatierapport
        \item Aanpassen van project en conclusie: 2 weken.\\ \emph{Deliverable} -> Aangepaste proof of concept, eindrapport en conclusie
\end{itemize}

%---------- Verwachte resultaten ----------------------------------------------
\section{Verwacht resultaat, conclusie}%
\label{sec:verwachte_resultaten}
Ik verwacht natuurlijk dat Deltalex geniet van mijn oplossing en dat alle vennoten hun tijd kunnen investeren in dingen die écht belangrijk zijn. Daarmee vermijden ze repetitieve taken die heel tijdrovend zijn. Ook zal de stress en frustratie op de werkvloer afnemen. Dat zal resulteren in een aangename werkomgeving waar iedereen zich productief bezighoudt en zichzelf amuseert in hun werk.

Naar de praktische kant toe verwacht ik drie dingen:
\begin{itemize}
        \item \emph{Requirementsanalyse}
        \item \emph{Ontwerp en implementatie van gekozen technologieën}
        \item \emph{Evaluatie en validatie}
\end{itemize}

Ik hoop ook dat mijn bachelorproef kan motiveren waarom een kantoor (in dit geval advocatenkantoor) voordeel zou kunnen halen uit specifieke implementaties, die kunnen helpen aan bepaalde nicheproblemen, zonder een volledig nieuw systeem te implementeren. Een nieuw systeem moet aangekocht, aangeleerd en aangepast worden. Dit is een kwestie van hoge kosten, potentieel lange gewenning en mogelijke datamigratieproblemen. 


\printbibliography[heading=bibintoc]

\end{document}
