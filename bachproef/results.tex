\chapter{Resultaten en reflectie}
Een reflectie verloopt heel goed aan de hand van auto-evaluatie d.m.v. vragen die men aan zichzelf stelt. 
Hieronder volgen een paar vragen die kunnen ontspringen wanneer men denkt aan automatisatie van workflows en repetitieve taken.\\ 

\textbf{Welke taken zijn er nu makkelijk te automatiseren?}
De taken die hier in deze bachelorproef beschreven worden gaan over:
\begin{itemize}
	\item \textbf{Opzoekingswerk verrichten in documentendatabases}
	\item \textbf{Opzoekingswerk verrichten in publieke databases}
	\item \textbf{Opstellen van documenten}
\end{itemize}
Een digitale assistent kan een advocaat helpen met deze taken te versnellen door tekst te genereren ter inspiratie en met relevante informatie uit een database. 
Het opstellen en gebruiken(versturen, indienen bij de rechtbank, ...) van deze documenten moet nog altijd gebeuren door de advocaat zelf. 
Een digitale assistent dient niet ter vervanging van een advocaat, maar wel als een tool die hun workflow kan versnellen. \\ 

\textbf{Hoeveel productiviteitswinst kan een advocaat realiseren bij deze taken?}
Het gebruik van een digitale assistent zal opzoekingswerk versnellen en dienen ter inspiratie. 
Om nog efficiënter te kunnen werken kan een advocaat technieken leren zoals sneltoetsen en tools gebruiken zoals browserextensies om hun workflow exponentieel te optimaliseren. 
Hoeveel productiviteitswinst er dan is, hangt af van advocaat tot advocaat. 
Sommigen zullen hun workflow bewaren zoals hij is en zijn er tevreden mee. 
Anderen zullen altijd op zoek zijn om deze te versnellen en te optimaliseren en kunnen zich hier ook actief mee bezighouden. 
Indien men de tijd wil vrijmaken om hier onderzoek naar te verrichten, zal er over de tijd een gestage stijging 
zijn in de snelheid van hun workflow door het gebruik van technieken en digitale assistenten. \\ 

\textbf{Waarmee moet men rekening houden bij de ontwikkeling van een digitale assistent?}
Het belangrijkste aspect bij de ontwikkeling van een digitale assistent bij een advocatenkantoor is waarschijnlijk dataveiligheid. 
Er mag absoluut, in geen enkele instantie twijfel ontstaan bij de gebruiker of bij de cliënt dat er vertrouwelijke data op het spel staat.  
Een dergelijk voorval zou potentieel destructief zijn voor de klantenrelaties en het imago van het advocatenkantoor. 
Andere belangrijke aspecten zijn gebruikersvriendelijkheid, snelheid van tokengeneratie en de kwaliteit van de gegenereerde antwoorden.\\ 

\textbf{Wat zijn de voor- en nadelen van een automatie via chatbots?}
De voordelen van chatbots zijn de snelheid van antwoorden, de bijna menselijke manier van interactie en de bron van inspiratie voor het opstellen van documenten. 
Chatbots zoals ChatGPT zijn de laatste jaren aan een gigantische opmars bezig en hun functionaliteiten worden alleen maar beter dankzij de gigantisch hoge graad van evolutie in de technologische sector. 
Nadelen van chatbots kunnen dingen zijn zoals hallucinatie(generatie van incorrecte antwoorden), grammaticale repetitie, ... 
Het is altijd aangeraden om hun output te valideren i.p.v. er blind op te vertrouwen. 
