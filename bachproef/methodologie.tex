%%=============================================================================
%% Methodologie
%%=============================================================================

\chapter{\IfLanguageName{dutch}{Methodologie}{Methodology}}%
\label{ch:methodologie}

%% TODO: In dit hoofstuk geef je een korte toelichting over hoe je te werk bent
%% gegaan. Verdeel je onderzoek in grote fasen, en licht in elke fase toe wat
%% de doelstelling was, welke deliverables daar uit gekomen zijn, en welke
%% onderzoeksmethoden je daarbij toegepast hebt. Verantwoord waarom je
%% op deze manier te werk gegaan bent.
%% 
%% Voorbeelden van zulke fasen zijn: literatuurstudie, opstellen van een
%% requirements-analyse, opstellen long-list (bij vergelijkende studie),
%% selectie van geschikte tools (bij vergelijkende studie, "short-list"),
%% opzetten testopstelling/PoC, uitvoeren testen en verzamelen
%% van resultaten, analyse van resultaten, ...
%%
%% !!!!! LET OP !!!!!
%%
%% Het is uitdrukkelijk NIET de bedoeling dat je het grootste deel van de corpus
%% van je bachelorproef in dit hoofstuk verwerkt! Dit hoofdstuk is eerder een
%% kort overzicht van je plan van aanpak.
%%
%% Maak voor elke fase (behalve het literatuuronderzoek) een NIEUW HOOFDSTUK aan
%% en geef het een gepaste titel.

In dit hoofdstuk volgt wat er precies zal gebeuren om een RAG client app te bouwen die toepasbaar kan zijn voor grote datasets zoals die van Deltalex Advocaten. 
De volgende hoofdstukken zullen uitgebreid beschreven, onderzocht en uitgetest worden. \\

Het onderzoek naar oplossingen is parallel gevoerd samen met het opbouwen van een stappenplan om een werkende generieke digitale assistent te bekomen. 
Eerst worden de stappen van het onderzoek overlopen, dan gaan we naadloos over naar onderzoek over welke stappen en technologieën er aan bod komen in de ontwikkeling van een digitale assistent. 

\begin{enumerate}
    \item \textbf{Onderzoek: Requirements-analyse}
    \item \textbf{Onderzoek: Literatuurstudie}
    \item \textbf{Onderzoek: Haalbaarheid en dataveiligheid} 

    \item \textbf{Development: Documentdatabase}
    \item \textbf{Development: Context-aware chunking van verzamelde data}
    \item \textbf{Development: Integreren van Ollama met documentdatabase}
    \item \textbf{Development: Implementeren van een chatfrontend}
    \item \textbf{Development: Opzetten van lokale omgeving en componenten verbinden}
    \item \textbf{Development: Optimaliseren en testen}  

    \item \textbf{Resultaten}
\end{enumerate} 

%%Op het einde van deze reis is de uitkomst een digitale assistent die kennis heeft van alle dossierdata van een hypothetisch advocatenkantoor. 
%%Deze zal advocaten helpen met het opzoeken van informatie, opstellen van brieven, communicatie en dergelijke. 


