%%=============================================================================
%% Methodologie
%%=============================================================================

\chapter{\IfLanguageName{dutch}{Methodologie}{Methodology}}%
\label{ch:methodologie}

%% TODO: In dit hoofstuk geef je een korte toelichting over hoe je te werk bent
%% gegaan. Verdeel je onderzoek in grote fasen, en licht in elke fase toe wat
%% de doelstelling was, welke deliverables daar uit gekomen zijn, en welke
%% onderzoeksmethoden je daarbij toegepast hebt. Verantwoord waarom je
%% op deze manier te werk gegaan bent.
%% 
%% Voorbeelden van zulke fasen zijn: literatuurstudie, opstellen van een
%% requirements-analyse, opstellen long-list (bij vergelijkende studie),
%% selectie van geschikte tools (bij vergelijkende studie, "short-list"),
%% opzetten testopstelling/PoC, uitvoeren testen en verzamelen
%% van resultaten, analyse van resultaten, ...
%%
%% !!!!! LET OP !!!!!
%%
%% Het is uitdrukkelijk NIET de bedoeling dat je het grootste deel van de corpus
%% van je bachelorproef in dit hoofstuk verwerkt! Dit hoofdstuk is eerder een
%% kort overzicht van je plan van aanpak.
%%
%% Maak voor elke fase (behalve het literatuuronderzoek) een NIEUW HOOFDSTUK aan
%% en geef het een gepaste titel.
Dit onderzoek is verdeeld in drie grote fasen:

\begin{enumerate}
	\item Een grondig onderzoek naar de beschikbare technologieën die geschikt zijn voor onze assistent en die vooral ook compatibel zijn met elkaar.
	      Dit omvat onderzoek naar verschillende databases, frameworks en libraries.
	      Het doel van deze fase is om een solide technologische basis te creëren voor de ontwikkeling van de digitale assistent.
	\item Write-up van de implementatie. Hier wordt stap per stap overlopen hoe Deltalex Chat tot leven zal komen.
	      Alle besproken technologieën worden samengevoegd tot een werkend geheel.
	\item Evaluatie en testing van Deltalex Chat bij een advocaat.
	      Dit hoofdstuk documenteert de effectiviteit van de implementatie en de impact ervan op de workload in een real-life omgeving.
\end{enumerate}

In deze drie fasen zal gepoogd worden om het volledige proces zo specifiek mogelijk in kaart te brengen voor de lezer van deze bachelorproef. 

%%Op het einde van deze reis is de uitkomst een digitale assistent die kennis heeft van alle dossierdata van een hypothetisch advocatenkantoor. 
%%Deze zal advocaten helpen met het opzoeken van informatie, opstellen van brieven, communicatie en dergelijke. 


