\chapter{Opzetten van lokale omgeving en componenten verbinden}
Nu het tijd is om alles samen te brengen, volgen we de volgende stappen om alles lokaal te draaien:

\section{Frontend voorbereiden}
Onze Next.js frontend stellen we in om HTTP requests en responses te faciliteren van en naar ons Langchain systeem. 
Dit zal vrij eenvoudig verlopen omdat veel implementatie al voor ons gedaan is in het template project. 

\section{RAG systeem voorbereiden}
Langchain is een framework dat naast het controleren van onze modellen en vector stores ook een REST API host, die gecalled kan worden door onze chat-frontend. 
Op deze manier houden we onze frontend en backend gescheiden van elkaar en verbinden we ze enkel over HTTP netwerkverkeer. 
We kunnen dus op de machine van iedere advocaat een frontend installeren die veilig communiceert met één interne backend. 

Nu deze twee met elkaar kunnen communiceren, zal het systeem intensief getest moeten worden op zwaar verkeer, incorrecte input, foute antwoorden van ons LLM, enz. 

