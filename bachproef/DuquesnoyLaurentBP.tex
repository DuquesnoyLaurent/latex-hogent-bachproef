%===============================================================================
% LaTeX sjabloon voor de bachelorproef toegepaste informatica aan HOGENT
% Meer info op https://github.com/HoGentTIN/latex-hogent-report
%===============================================================================

\documentclass[dutch,dit,thesis]{hogentreport}

% TODO:
% - If necessary, replace the option `dit`' with your own department!
%   Valid entries are dbo, dbt, dgz, dit, dlo, dog, dsa, soa
% - If you write your thesis in English (remark: only possible after getting
%   explicit approval!), remove the option "dutch," or replace with "english".

\usepackage{lipsum} % For blind text, can be removed after adding actual content
\usepackage{glossaries}

%% Pictures to include in the text can be put in the graphics/ folder
\graphicspath{{graphics/}}
\makeglossaries


%%---------- Document metadata -------------------------------------------------
\author{Laurent Duquesnoy}
\supervisor{Dhr. J. Claes}
\cosupervisor{Mr. S. Dewolf}
\title%
    {Optimalisatie van de administratieve workflow in juridische kantoren: automatisering van repetitieve taken}
\academicyear{\advance\year by -1 \the\year--\advance\year by 1 \the\year}
\examperiod{1}
\degreesought{\IfLanguageName{dutch}{Professionele bachelor in de toegepaste informatica}{Bachelor of applied computer science}}
\partialthesis{false} %% To display 'in partial fulfilment'
%\institution{Internshipcompany BVBA.}

%% Add global exceptions to the hyphenation here
\hyphenation{back-slash}

%% The bibliography (style and settings are  found in hogentthesis.cls)
\addbibresource{bachproef.bib}            %% Bibliography file
\addbibresource{../voorstel/voorstel.bib} %% Bibliography research proposal
\defbibheading{bibempty}{}

%% Prevent empty pages for right-handed chapter starts in twoside mode
\renewcommand{\cleardoublepage}{\clearpage}

\renewcommand{\arraystretch}{1.2}

%% Content starts here.
\begin{document}

%---------- Front matter -------------------------------------------------------

\frontmatter

\hypersetup{pageanchor=false} %% Disable page numbering references
%% Render a Dutch outer title page if the main language is English
\IfLanguageName{english}{%
	%% If necessary, information can be changed here
	\degreesought{Professionele Bachelor toegepaste informatica}%
	\begin{otherlanguage}{dutch}%
		\maketitle%
	\end{otherlanguage}%
}{}

%% Generates title page content
\maketitle
\hypersetup{pageanchor=true}

%%=============================================================================
%% Voorwoord
%%=============================================================================

\chapter*{\IfLanguageName{dutch}{Woord vooraf}{Preface}}%

Beste lezer, als u dit leest hebt u een kopie bemachtigd van mijn bachelorthesis. 
Ik zou eerst en vooral mijn promotor, de heer Jan Claes van HoGent en mijn co-promotor, 
Stijn Dewolf van Deltalex van harte willen bedanken voor de goede ondersteuning en behulpzaamheid tijdens dit project. 
Dit is een thesis geschreven door een student die een passie heeft in het gebied van Informatietechnologie en er al heel zijn leven mee bezig is. 
Deze passie heeft mij toegelaten om te denken als een problem solver en problemen rond mij te zoeken, vast te stellen en vervolgens te proberen er een oplossing voor te bedenken. \\

Het probleem dat ik in deze thesis zal bestuderen is het probleem van repetitie op de werkvloer, meer specifiek in een advocatenkantoor genaamd Deltalex. 
Ik merkte op dat er daar heel veel taken gebeuren die heel repetitief zijn in aard. 
Als programmeur heb ik geleerd dat automatisatie een krachtig denkpatroon is, dus dacht ik dan ook om dit probleem eens onder de loep te nemen. 
Mogelijke oplossingen die mij te binnen schoten waren plugins in document editors (b.v. Word), custom document macro's e.d. \\

Na wat feedback van mijn promotor ben ik op het idee gekomen om een digitale assistent te bouwen om de advocaten in allerlei administratieve taken bij te staan. 
Nu is het probleem dat er bij advocaten heel veel gewerkt wordt met confidentiële data. 
Het is dus imperatief dat de data van cliënten nooit de veilige omgeving van Deltalex verlaat. 
Om deze regel te respecteren heb ik een digitale assistent gebouwd die volledig lokaal op een server kan draaien. \\

De bedoeling van deze thesis is dus om u, de lezer, zo goed mogelijk te informeren over mijn afgelegde route in het ontwikkelen van deze assistent. 
Ik hoop dan ook om u zo veel mogelijk bij te kunnen leren tijdens dit avontuur van automatisatie en de verkenning van Artificiële intelligentie. 

%%=============================================================================
%% Samenvatting
%%=============================================================================

% TODO: De "abstract" of samenvatting is een kernachtige (~ 1 blz. voor een
% thesis) synthese van het document.
%
% Een goede abstract biedt een kernachtig antwoord op volgende vragen:
%
% 1. Waarover gaat de bachelorproef?
% 2. Waarom heb je er over geschreven?
% 3. Hoe heb je het onderzoek uitgevoerd?
% 4. Wat waren de resultaten? Wat blijkt uit je onderzoek?
% 5. Wat betekenen je resultaten? Wat is de relevantie voor het werkveld?
%
% Daarom bestaat een abstract uit volgende componenten:
%
% - inleiding + kaderen thema
% - probleemstelling
% - (centrale) onderzoeksvraag
% - onderzoeksdoelstelling
% - methodologie
% - resultaten (beperk tot de belangrijkste, relevant voor de onderzoeksvraag)
% - conclusies, aanbevelingen, beperkingen
%
% LET OP! Een samenvatting is GEEN voorwoord!

%%---------- Nederlandse samenvatting -----------------------------------------
%
% TODO: Als je je bachelorproef in het Engels schrijft, moet je eerst een
% Nederlandse samenvatting invoegen. Haal daarvoor onderstaande code uit
% commentaar.
% Wie zijn bachelorproef in het Nederlands schrijft, kan dit negeren, de inhoud
% wordt niet in het document ingevoegd.

\IfLanguageName{english}{%
\selectlanguage{dutch}
\chapter*{Samenvatting}
\selectlanguage{english}
}{}

%%---------- Samenvatting -----------------------------------------------------
% De samenvatting in de hoofdtaal van het document

\chapter*{\IfLanguageName{dutch}{Samenvatting}{Abstract}}

In kantoren met een grote administratieve workload, zoals een advocatenkantoor, bestaat er een grote kans dat er een heel groot aantal aan documenten 
in het archief en verschillende databanken resideert.  Het is niet altijd makkelijk voor advocaten en medewerkers om zich aan te passen aan het snel evoluerende klimaat van informatietechnologie omdat zij zich vooral 
focussen op het (ook snel evoluerende) rechtssysteem. Zodoende stagneert (of evolueert) de kwaliteit van de ondersteuning, maar de productiviteit daarentegen kan verlagen.  \\

Daarom kan het handig zijn voor dergelijke kantoren om te investeren in een systeem dat dienst kan doen als een digitale assistent. 
Deze bachelorproef gaat over de implementatie van dergelijk systeem. 
Hij is verdeeld in in verschillende stappen. 
Deze stappen zullen de rode draad vormen in deze proef en gaan over een analyse van het kantoor en mogelijke bottlenecks, het vergaren van trainingdata, het bouwen van een server en interface
voor advocaten om te werken met een tool die uiteindelijk hun productiviteit zal verhogen. \\  

Ook zal er analyse gedaan worden naar welke technologieën er voor handen zijn en welke er het best passen bij de toepassingen. 
Bijvoorbeeld zal er onderzocht worden welk type database het snelst kan omgaan met de data die gebruikt wordt, 
welke programma's er passend zijn voor implementatie, ... 
Ook wordt via de eerste analyse bekeken welke toepassingen haalbaar zijn in het tijdskader van 12 weken. \\

Achteraf volgt een korte analyse, die stukken van het Business Acceptance Model gebruikt om te schetsen wat de impact is.



%---------- Inhoud, lijst figuren, ... -----------------------------------------

\tableofcontents
\listoffigures
% In a list of figures, the complete caption will be included. To prevent this,
% ALWAYS add a short description in the caption!
%
%  \caption[short description]{elaborate description}
%
% If you do, only the short description will be used in the list of figures

\newglossaryentry{LLM}{
	name=LLM,
	description={Large Language Model: Een taalmodel dat uitblinkt in het begrijpen en genereren van generieke taal op basis van een grote set trainingdata.}
}
\newglossaryentry{ERP}{
	name=ERP,
	description={Enterprise Resource Planning: Een systeem dat een bedrijf helpt met het organiseren van vrijwel elk bedrijfsaspect, van Human Resources, Supply Chain Management tot Finance.}
}
\newglossaryentry{CRM}{
	name=CRM,
	description={Customer Relationship Management: Een systeem om alle interacties met bestaande en toekomstige klanten te beheren}
}
\newglossaryentry{NLP}{
	name=NLP,
	description={Natural Language Processing: Het begrijpen en verwerken van natuurlijke taal door een computer}
}
\newglossaryentry{framework}{
	name=Framework,
	description={een structuur die dienst doet als een soort skelet, gemaakt om iets te bevatten of ondersteunen. Denk aan een stelling rond een bouwwerf}
}
\newglossaryentry{feature}{
	name=Feature,
	description={Een individueel, meetbaar kenmerk van data. Er zijn twee veelgebruikte soorten: categoriegebaseerd(geslacht, kleur, postcode) en numeriek(leeftijd, gewicht, inkomen)}
}
\newglossaryentry{Feature extraction}{
	name={Feature Extraction},
	description={Een voorbarige stap in machine learning die ruwe data transformeert in een effectievere set van features.}
}
\newglossaryentry{Machine Learning}{
	name={Machine Learning},
	description={De wetenschap die algoritmes bestudeert en ontwikkelt die kunnen leren van data en op basis van die data taken kunnen uitvoeren zonder expliciete instructies. Denk maar aan stemherkenning.}
}
\newglossaryentry{technologiestack}{
	name=Technologiestack,
	description={Een set technologieën die samen een geheel bouwen, veel bedrijven werken bijvoorbeeld met de Microsoft Stack(Office, Azure, ...) en zijn hier ook aan gebonden. }
}


\printglossary[]

% If you included tables and/or source code listings, uncomment the appropriate
% lines.
%\listoftables
%\listoflistings

% Als je een lijst van afkortingen of termen wil toevoegen, dan hoort die
% hier thuis. Gebruik bijvoorbeeld de ``glossaries'' package.
% https://www.overleaf.com/learn/latex/Glossaries

%---------- Kern ---------------------------------------------------------------

\mainmatter{}

% De eerste hoofdstukken van een bachelorproef zijn meestal een inleiding op
% het onderwerp, literatuurstudie en verantwoording methodologie.
% Aarzel niet om een meer beschrijvende titel aan deze hoofdstukken te geven of
% om bijvoorbeeld de inleiding en/of stand van zaken over meerdere hoofdstukken
% te verspreiden!

%%=============================================================================
%% Inleiding
%%=============================================================================

\chapter{\IfLanguageName{dutch}{Inleiding}{Introduction}}%
Deze bachelorproef handelt over het verbeteren van de efficiëntie van administratieve taken bij advocatenkantoren.
De natuur van de taken die deze beoogt te verbeteren zijn veelal repetitief in aard.
Dit onderzoek ontspringt zich in de observatie dat bepaalde taken die uitgevoerd worden frustrerend en traag zijn.
Een technologisch onderzoek heeft een hoge kans om de tijd die deze taken innemen vrij te maken en mogelijk te herinvesteren in taken die meer uitdagend zijn en variëren in aard.

\section{\IfLanguageName{dutch}{Probleemstelling}{Problem Statement}}%
\label{sec:probleemstelling}

Als we kijken bij de administratieve backend in Deltalex advocaten en observeren hoe bepaalde taken gebeuren, valt op dat veel taken die uitgevoerd worden repetitief en tijdrovend zijn.
Deze taken kunnen impact hebben op de productiviteit van medewerkers, de herhalende aard van de taken vergroot
ook de kans op fouten die makkelijk overzien worden maar een grote impact kunnen hebben op het kantoor.
Denk maar aan foute looncalculaties, miscommunicatie, spelfouten, \dots
Natuurlijk is het niet eenvoudig om een automatisatie toe te passen op een heel specifiek punt, deze bachelorproef zal deels dienen als een Proof Of Concept om de haalbaarheid van
dergelijke tools in een geavanceerde kantooromgeving te illustreren.

\section{\IfLanguageName{dutch}{Onderzoeksvraag}{Research question}}%
\label{sec:onderzoeksvraag}

Uit de bovenstaande passage ontspringt de vraag:
"Bestaat er een mogelijkheid om repetitieve taken te automatiseren?
Kan een dergelijke toepassing de productiviteit van een advocaat positief beïnvloeden?
Hoe haalbaar is een dergelijke toepassing?
Kan deze dienen als vervanging van de manuele uitvoer van zulke taken?".

\section{\IfLanguageName{dutch}{Onderzoeksdoelstelling}{Research objective}}%
\label{sec:onderzoeksdoelstelling}
Wat is het beoogde resultaat van je bachelorproef?
Wat zijn de criteria voor succes? Beschrijf die zo concreet mogelijk. Gaat het bv. om een proof-of-concept, een prototype, een verslag met aanbevelingen, een vergelijkende studie, enz.

Het beoogde resultaat van deze bachelorproef kan opgedeeld worden in zes delen:
\begin{itemize}
	\item Het versnellen en verbeteren van de workflow van advocaten die werkzaam zijn bij Deltalex
    \item Het analyseren van de data (dossiers, communicatie, documenten) die gebruikt kan worden voor optimalisatie
    \item Naast een digitale assistent, een potentiële snelle, geoptimaliseerde zoekmachine voor hun documentendatabase
    \item De haalbaarheid van deze features in kaart brengen
    \item Aantonen via onderzoek of een dergelijke implementatie concrete voordelen heeft op de manuele executie van taken
    \item Proberen de implementatie zo nauw mogelijk te integreren met bestaande software
\end{itemize}


\section{\IfLanguageName{dutch}{Opzet van deze bachelorproef}{Structure of this bachelor thesis}}%
\label{sec:opzet-bachelorproef}

% Het is gebruikelijk aan het einde van de inleiding een overzicht te
% geven van de opbouw van de rest van de tekst. Deze sectie bevat al een aanzet
% die je kan aanvullen/aanpassen in functie van je eigen tekst.

De rest van deze bachelorproef is als volgt opgebouwd:

In Hoofdstuk~\ref{ch:stand-van-zaken} wordt een overzicht gegeven van de stand van zaken binnen het onderzoeksdomein, op basis van een literatuurstudie.

In Hoofdstuk~\ref{ch:methodologie} wordt de methodologie toegelicht en worden de gebruikte onderzoekstechnieken besproken om een antwoord te kunnen formuleren op de onderzoeksvragen.

% TODO: Vul hier aan voor je eigen hoofstukken, één of twee zinnen per hoofdstuk

In Hoofdstuk~\ref{ch:conclusie}, tenslotte, wordt de conclusie gegeven en een antwoord geformuleerd op de onderzoeksvragen. Daarbij wordt ook een aanzet gegeven voor toekomstig onderzoek binnen dit domein.


\chapter{\IfLanguageName{dutch}{Stand van zaken}{State of the art}}%
\label{ch:stand-van-zaken}

% Tip: Begin elk hoofdstuk met een paragraaf inleiding die beschrijft hoe
% dit hoofdstuk past binnen het geheel van de bachelorproef. Geef in het
% bijzonder aan wat de link is met het vorige en volgende hoofdstuk.

% Pas na deze inleidende paragraaf komt de eerste sectiehoofding.

Dit hoofdstuk bevat een literatuurstudie op basis van een paar topics die relevant zijn met onze onderzoeksvraag. Het is verdeeld in verschillende onderdelen waarvan de eerste drie een probleem
beschrijven dat geobserveerd werd door enerzijds de advocaten zelf en anderzijds door uitgebreide research van relevante papers en artikels op het web.

De volgende drie onderdelen handelen over
mogelijke obstakels en belangrijke principes die men in het achterhoofd moet houden wanneer men een tool ontwikkelt die beantwoordt aan de probleemstelling.\\

Er wordt hier en daar ook al een beetje kennis gemaakt met enkele technische aspecten (zoals LLM's) die fundamentele bouwstenen kunnen vormen voor een dergelijke implementatie.

Het laatste hoofdstuk handelt over research omtrent het Technology Acceptance Model. Hier wordt onderbouwd hoe efficiënt het kan zijn en hoe een ontwikkelaar er het meest efficiënt gebruik
van kan maken.


\section{Inefficiënt documentmanagement en repetitie}
In een advocatenkantoor is het vanzelfsprekend dat er heel veel administratie aan bod komt.
Denk maar aan het opstellen van dagvaardingen,
ingebrekestellingen, e-mails, aangetekende brieven en andere communicatie.

Natuurlijk dringt zich dit op dat deze taken veel tijd in beslag kunnen nemen. Naast een administratief spectrum gaat het ook over verdediging van cliënten, pleiten en ander rechtbankwerk.

\newpage
Moest er bij al die bestaande taken dan nog extra repetitief bureauwerk aan te pas komen, zoals copy-pasten van informatie, handmatig opzoekwerk en dergelijke,
kan dit een groot obstakel vormen naar productiviteit toe.

Hiervoor onderstaande citatie ter illustratie, deze noemt een aantal cijfers op die voorgaande probleemstelling onderbouwen.
\begin{displayquote}
	\textit{"How much time is spent in repetitive tasks in the workplace? Today the typical office worker spends 10\% of their time on manual data entry into business applications,
		such as the ERP system, CRM or spreadsheets. In total, they spend over 50\% of work time creating or updating documents, eg. PDFs, spreadsheets or word documents \autocite{Workfellow}."}
\end{displayquote}

Dit is een generieke citatie die handelt over PDFs, spreadsheets en dergelijke, dit is dus toepasbaar in eender welke kantooromgeving, dus zeker ook in een advocatenkantoor.

De conclusie en probleemstelling is hier dat repetitie een killer is voor productiviteit en planning.

Neem nu als voorbeeld copy-pasten van data of het organiseren van bestanden: hieronder volgt een citatie met enkele cijfers die illustreert hoeveel tijd een gemiddelde administratieve medewerker
spendeert aan bepaalde taken. Onderstaande is een extensie van bovenstaande citatie.

\begin{displayquote}
	\begin{itemize}
		\item \emph {"A typical office worker spends 3 hours working on spreadsheets each week, for example in Microsoft Excel or Google Sheets. "}\autocite{Workfellow}
		\item \emph {"A typical office worker spends almost 2 1/2 hours in business communications or email applications, such as Outlook."}\autocite{Workfellow}
		\item \emph {"A typical office worker spends over 1 1/2 hours each week searching and organizing files, for example in the shared file service such as Sharepoint or Google Drive."}\autocite{Workfellow}
		\item \emph {"A typical office worker spends 1 1/2 hours each week copy-pasting or manually entering data into business applications, such as the ERP or CRM."} \autocite{Workfellow}
	\end{itemize}
\end{displayquote}

Hoe kunnen we repetitie wegwerken? Dit probleem kan ontstaan door verschillende oorzaken, zoals een te groot (of ongeorganiseerd) archief van documenten of verschillende obstakels en valkuilen
tijdens het researchen. Laten we deze twee eens onder de loep nemen.
\newpage


\section{Het managen van grote hoeveelheden documenten}
\subsection{Probleemstelling}
Doorheen de jaren wordt er een heel groot archief aan documenten opgebouwd, denk maar aan communicatie, dossiers, facturen enz.
Het doorzoeken en onderhouden van dergelijk archief kan enerzijds heel complex worden en anderzijds onnodig veel tijd in beslag nemen.
Zodoende kan de productiviteit van een kantoor in het nauw gedreven worden.

Hoe (her)organiseer je zo een archief? Hoe kan je er het best in zoeken? Hoe kan er tijd bespaard worden?

\subsection{Mogelijke hulpmiddelen en oplossingen}
In eerste instantie is het opzoeken in een archief belangrijker op korte termijn dan een volledige herstructurering. Een archief is al lang niet meer een oude metalen kast die in de kelder van een
kantoor staat. Dit is bij de meeste bedrijven al lang geëveolueerd naar een digitale dataset. Moet deze dan constant manueel doorzocht worden?

Het opzoeken van documenten in een digitaal archief is een mooi voorbeeld om te vergelijken tussen menselijke en machinale zoekmethodes.
Het kan immers parallel gesteld worden met een online zoekmachine. Volgende citatie geeft ons een korte introductie over zoekmachines.

\begin{displayquote}
	\textit{"Search engines have been with us for several decades as an integral part of our digital life.
		We are casually searching over billions of web pages to retrieve and share information from various resources.
		While humans are very good at conversational context and background knowledge which helps them to deal with intrinsic ambiguity of words,
		it may not be true in case of search engines."} \autocite{MediumSemanticSearch}
\end{displayquote}

Het artikel gaat dan verder en zegt dat er iets bestaat als 'Semantisch zoeken'. Semantisch zoeken zal niet alleen naar de letterlijk ingevoerde woorden van een zoekterm kijken,
maar zal ook de betekenis, context en het algemeen onderwerp er uit proberen extraheren. Dit helpt om relevante resultaten te bieden en kan bereikt worden via Natural Language Processing, oftewel NLP.

Zoekmachines, en zeker degene uit de "primitieve" generatie, gebruikten echter niet altijd een semantische methodiek:

\begin{displayquote}
	\textit{"At first, search engines were lexical: the search engine looked for literal matches of the query words,
		without understanding of the query’s meaning and only returning links that contained the exact query."} \autocite{MediumSemanticSearch}
\end{displayquote}

Lexicaal zoeken is (relatief gezien) simpel te implementeren, dus een "primitief" concept dat letterlijke matches zoekt van de query in een bepaalde dataset.
Semantisch zoeken is daarentegen wat complexer in opbouw.

\begin{displayquote}
	\textit{"On the other hand, “Semantic Search” can simplify query building, because it is supported by automated natural language processing programs
		i.e. using Latent Semantic Indexing — a concept that search engines use to discover how a keyword and content work together to mean the same thing."} \autocite{MediumSemanticSearch}
\end{displayquote}

Semantisch zoeken is hier vooral een toepasbare topic omdat het erg afhankelijk is van NLP, een van de belangrijkste topics van deze paper.

Nu, een zoekmachine is dus een polyvalente tool, die kan worden ingezet op verschillende, variërende datasets.
Dit kan dus bijvoorbeeld een volledig archief zijn van documenten in een advocatenkantoor.
Het implementeren van een semantisch zoekalgoritme kan een oplossing bieden op het inefficiënt zoeken in dit archief.

Waarom zouden we hier geen lexicaal zoekalgoritme toepassen?
Omdat contextueel zoeken is soms iets heel belangrijk kan zijn, het kan perfect zijn dat men zoekt voor iets dat niet letterlijk (alle) zoektermen kan bevatten.

Een lexicaal algoritme is dus niet de beste oplossing omdat het tekort kan schieten in bepaalde gevallen. Medium illustreert dit ook aan de hand van LSI:

\begin{displayquote}
	\textit{"In brief, LSI(Latent Semantic Index) does not require an exact match to return useful results.
		Where a plain keyword search will fail if there is no exact match,
		LSI will often return relevant documents that don’t contain the keyword at all." }\autocite{MediumSemanticSearch}
\end{displayquote}

\section{Obstakels in research van rechtszaken}
\subsection{Probleemstelling}
Bij een (oppervlakkige) verkennende audit bij advocatenkantoor Deltalex viel op dat er veel tijd wordt gespendeerd aan het opzoeken van dossiergerelateerde informatie.
Denk hierbij maar aan contactgegevens (ondernemingsnummer, adres, ...) van een cliënt, dossierdata, technische specificaties, enz.
In het vorige deel werd al besproken dat zoekmachines ons hiermee kunnen helpen. Deze kunnen documenten teruggeven die een goeie match zijn met onze zoekcriteria.
Maar hoe giet dit met in een antwoord van bv. een digitale assistent?

\subsection{Mogelijke hulpmiddelen en oplossingen}
Voor een advocaat kan het formuleren van een antwoord (gebaseerd op een zoekresultaat) een lang en repetitief proces vormen indien manueel uitgevoerd.
Gelukkig kunnen hier technieken zoals RAG (Retrieval-Augmented Generation) een grote hulp bij bewijzen.

RAG (Retrieval Augmented Generation) is een van de grootste use cases voor LLM's(Large Language Models) en is een framework dat de fundamentele trainingsdata van een Large Language Model
combineert met bestaande data in een database. Zodoende kan een model gegronde antwoorden geven, gebaseerd op betrouwbare en relevante informatie.

Een quote van Medium legt uit hoe RAG werkt in grote lijnen:

\begin{displayquote}
	\textit{"A typical RAG process, as pictured below, has an LLM, a collection of enterprise documents, and supporting infrastructure to improve information retrieval and answer construction.
		The RAG pipeline looks at the database for concepts and data that seem similar to the question being asked, extracts the data from a vector database and reformulates the data into
		an answer that is tailored to the question asked. This makes RAG a powerful tool for companies looking to harness their existing data repositories for enhanced decision-making
		and information access."} \autocite{MediumRAG}
	\begin{figure}[h]
		\includegraphics[width=\textwidth]{RAG.png}
		\centering
	\end{figure}
\end{displayquote}
\newpage

RAG kan hier dienst doen als een mogelijke plaatsvervanger voor manuele research en het opstellen van communicatiedocumenten (zoals aangetekende brieven, invorderingen, ...). Deze optie zal in
een later stadium van deze bachelorproef geëvealueerd worden. Advocatuur is een heel toepasselijk gebied voor RAG:

\begin{displayquote}
	\textit{"Practically, RAG is likely preferable in environments like
		legal, customer service, and financial services where the ability to
		dynamically pull vast amounts of up-to-date data enables the most accurate and comprehensive responses."} \autocite{MediumRAG}
\end{displayquote}

\section{Het automatiseren van administratieve taken}
\subsection{Probleemstelling}
Het manueel en persoonlijk afhandelen van 'simpele' administratieve taken door een advocaat zelf kan heel veel tijdverlies veroorzaken.
Een digitale assistent kan bepaalde zaken overnemen, zoals het inplannen van consultaties, agendabeheer en routinecommunicatie. \\
In hoeverre kan een systeem dit overnemen? Is dit mogelijk?

De eerste zaak waar een programmeur zijn hoofd over kan breken:
hoe kunnen we een assistent toegang (lezen en schrijven) geven tot de persoonlijke agenda's van advocaten, programma's en persoonlijke omgevingen?

Wat als er iets fout gaat en heel de agenda wordt overschreven met nutteloze data?
Wat als afspraken bevestigd worden maar niet ingepland?

Het is ook niet echt ethisch verantwoord als een cliënt een som betaalt voor persoonlijke aandacht en communicatie en dan gegenereerde content voorgeschoteld krijgt.
Aan de andere kant kan het ook zijn dat een advocaat in tijdsnood raakt en dan is het natuurlijk handig dat hij een snelle "content boost" ter zijn beschikking heeft.

\subsection{Mogelijke hulpmiddelen en oplossingen}
Een LLM (dat ook gebruikt kan worden bij voorgaande zaken) is hiervoor een ideale oplossing.
Het kan dienen ter inspiratie of ter generatie van een volledig document.
Natuurlijk moet een advocaat een persoonlijke behandeling verstrekken aan de cliënt dus deze optie wordt best alleen gebruikt in geval van nood.

Aan de andere kant kan een virtuele assistent wel instaan voor taken zoals agenda- en memobeheer.
Een dergelijke implementatie vereist integratie met bestaande software zoals Office365 (Microsoft) en ERP programma's.
Ook de analyse van persoonlijke communicatie van een advocaat en zijn cliënt zal ingebracht moeten worden in trainingsdata,
wat ons naadloos overbrengt naar de veiligheid van data tijdens het gebruik en training van digitale tools en assistenten.

\section{Privacy en veiligheid  van data}
In advocatenkantoren wordt op een dagelijkse basis omgegaan met vertrouwelijke informatie van cliënten.
Het is dan natuurlijk meer dan logisch dat de veiligheid van dergelijke informatie van elementair belang is.
Een citaat van "Legal buddies" staaft een paar protocollen om datasecurity en compliance in acht te nemen. 
Dit om met strikte dataregulaties zoals GDPR in de Europese Unie te volgen:

\begin{displayquote}
	\textit{"Proper protocols must be implemented to keep sensitive case data protected and align usage to regional regulations.}
	\begin{itemize}
		\item \emph{ When working with legal AI systems, law firms must implement security controls like data encryption, access management, network segmentation, and intrusion detection.}
		\item \emph{ Usage and data sharing policies should conform to relevant privacy laws and professional ethics rules around legal data confidentiality.}
		\item \emph{ Firms can request third-party audits of AI provider security infrastructure for assurance on protection mechanisms.}
		\item \emph{ Using on-premise AI options instead of cloud-based ones may better align with internal compliance rules and risk tolerance levels.}
		\item \emph{ Regional laws may also dictate data residency and cross-border transfer restrictions. Understanding jurisdictional nuances allows appropriate legal AI adoption.}
		\item \emph{ Overall, prudent security and compliance positioning is vital for law firms exploring innovative technologies like AI-powered legal assistants. Partnering with trusted, vetted providers also reduces risk exposure.}
	\end{itemize}

	\textit{With deliberate planning around privacy, ethics and regional legislation, firms can safely pursue AI efficiency gains."}\autocite{LegalBuddies}
\end{displayquote}

Bij verdere research blijkt het grootste risico dat data van cliënten (onveilig) over het internet gestuurd wordt of dat de data blijft plakken op de cloud server van een externe tool.
De betere en veiligste optie blijkt hier om oftewel lokaal of in private cloud te hosten. 
\\In het hoofdstuk "Methodologie" zullen verschillende technologieën besproken worden om een dergelijke setup te bereiken.

\section{Gebruiksgemak en aanpassing}
Een van de belangrijkste dingen in het inbrengen van nieuwe software in een bestaand kantoor is het gebruiksgemak en toegankelijkheid van de nieuwe programma's. 
Om deze zaken te garanderen, moeten er een paar principes nauwlettend gevolgd worden. 
De beste manier om een design te krijgen waar de gebruiker centraal in staat is het verstaan van de requirements, pijnpunten en workflows van advocaten. 
Dit is van elementair belang om een intuïtief systeem te bouwen dat naadloos integreert in hun workflow. Natuurlijk is dit niet het enige.
Laten we samen een paar belangrijke aspecten overlopen.

\subsection{Een intuïtieve interface}
De interface van een digitale assistent moet er overzichtelijk en netjes uitzien. 
Makkelijke navigatie is van elementair belang.
Advocaten moeten snel en makkelijk de features kunnen aanspreken zonder al te veel clutter en nutteloze toepassingen. 
De beste interface is er een die minimaal en ontworpen is met de gebruiker in het achterhoofd. 
Deze stelling kan onderbouwd worden met enkele designprincipes uit een artikel van Zefort.com:

\begin{displayquote}
	\textbf{Simplicity and clarity:}
	\textit{By reducing complexity and providing clear instructions, legal services become more user-friendly and facilitate better access to justice for all.}
	\autocite{Zefort}

	\textbf{Visual Hierarchy and Information Organization:}
	\textit{Prioritizing content through size, color, and contrast also aids in emphasizing key points and important details.}
	\autocite{Zefort}

	\textbf{Clear and Accessible Navigation:}
	\textit{A well-designed navigation system helps lawyers, clients, and other users locate the information they need with minimal effort.}
	\autocite{Zefort}
\end{displayquote}

\subsection{Natural language processing}
Om de leercurve van een digitale assistent te verlagen, moet deze in staat zijn om queries in natuurlijke taal te kunnen verwerken.
Dit laat advocaten en medewerkers toe om de assistent te benaderen op dezelfde manier als een menselijke collega. Meer over NLP in het hoofdstuk methodologie.

\subsection{Naadloze integratie}
Een dichte integratie met de bestaande software die gebruikt wordt is misschien wel één van de belangrijkste van de drie. Het succes van deze implementatie valt of staat met hoe makkelijk
een advocaat ze kan integreren in hun bestaande workflow. Dit elimineert manuele invoer van manuele data of het constant switchen tussen verschillende applicaties.

In conclusie: het succes van een digitale assistent (of in extensie bijna iedere digitale toepassing) hangt grotendeels af van de gebruikerservaring.
Als je de voorgaande drie zaken zult prioriteren in de ontwikkeling van een tool, kun je makkelijker het volledige potentieel ervan bereiken en zodoende een kwalitatieve, nuttige tool leveren.

\section{Technology Acceptance Model}
Een van de moeilijke dingen aan het implementeren van nieuwe software is controleren of het wel echt voordeel heeft op de workflow en of de mensen die ermee werken het accepteren. 
Daar heeft Fred Davis in '89 al een oplossing voor gevonden, namelijk het Technology Acceptance Model. 
Dit model focust op de bruikbaarheid en het gebruiksgemak van de applicaties via een paar componenten:

\begin{itemize}
	\item \textbf{Perceived Usefulness:} De graad in hoeverre een persoon gelooft dat een bepaald systeem hun productiviteit zal boosten
	\item \textbf{Perceived Ease of Use:} De graad in hoeverre een persoon gelooft dat een bepaald systeem makkelijk te gebruiken valt
	\item \textbf{Attitude towards Using:} De algemene attitude van een persoon tegenover het gebruik van een bepaald systeem, beïnvloed door bovenstaande
	\item \textbf{Behavioral Intention to Use:} De sterkte van de intentie van de gebruiker om een bepaald systeem te gebruiken
	\item \textbf{Actual System Use:} Het werkelijke gebruik van een bepaald systeem
\end{itemize}

Merk op dat deze componenten geobserveerd worden over verschillende tijdsperiodes. De laatste zal maar geobserveerd kunnen worden na de effectieve implementatie. 


%%=============================================================================
%% Methodologie
%%=============================================================================

\chapter{\IfLanguageName{dutch}{Methodologie}{Methodology}}%
\label{ch:methodologie}

%% TODO: In dit hoofstuk geef je een korte toelichting over hoe je te werk bent
%% gegaan. Verdeel je onderzoek in grote fasen, en licht in elke fase toe wat
%% de doelstelling was, welke deliverables daar uit gekomen zijn, en welke
%% onderzoeksmethoden je daarbij toegepast hebt. Verantwoord waarom je
%% op deze manier te werk gegaan bent.
%% 
%% Voorbeelden van zulke fasen zijn: literatuurstudie, opstellen van een
%% requirements-analyse, opstellen long-list (bij vergelijkende studie),
%% selectie van geschikte tools (bij vergelijkende studie, "short-list"),
%% opzetten testopstelling/PoC, uitvoeren testen en verzamelen
%% van resultaten, analyse van resultaten, ...
%%
%% !!!!! LET OP !!!!!
%%
%% Het is uitdrukkelijk NIET de bedoeling dat je het grootste deel van de corpus
%% van je bachelorproef in dit hoofstuk verwerkt! Dit hoofdstuk is eerder een
%% kort overzicht van je plan van aanpak.
%%
%% Maak voor elke fase (behalve het literatuuronderzoek) een NIEUW HOOFDSTUK aan
%% en geef het een gepaste titel.

In dit hoofdstuk volgt wat er precies zal gebeuren om een RAG client app te bouwen die toepasbaar kan zijn voor grote datasets zoals die van Deltalex Advocaten. 
De volgende hoofdstukken zullen uitgebreid beschreven, onderzocht en uitgetest worden. 
Deze komen overeen met een concreet stappenplan om een functionele, nuttige digitale assistent te bouwen op basis van documentdata. 

\begin{enumerate}
    \item \textbf{Opzetten van een database en verzamelen van alle documenten}
    \item \textbf{Context-aware chunking van verzamelde data}
    \item \textbf{Integreren van Ollama met documentdatabase}
    \item \textbf{Implementeren van een chatfrontend}
    \item \textbf{Optimaliseren en schalen van het systeem}
    \item \textbf{Opzetten van lokale omgeving}
    \item \textbf{Verbinden van chatfrontend met RAG systeem}
    \item \textbf{Optimaliseren en testen}
\end{enumerate} 

Op het einde van deze reis is de uitkomst een digitale assistent die kennis heeft van alle dossierdata van een hypothetisch advocatenkantoor. 
Deze zal advocaten helpen met het opzoeken van informatie, opstellen van brieven, communicatie en dergelijke. 



\chapter{Technologieën}
\label{ch:technologies}
Alle technologieën hieronder besproken zijn allemaal vrij simpel vervangbaar in een modulair systeem zoals LangChain.

\section{Backend}
\subsection{Databases}
Een van de belangrijkste componenten van een chatbot die gebouwd is voor specifieke doeleinden, is een plaats om data te kunnen opslaan.
Een gesprek met een RAG app kan schematisch voorgesteld worden als volgt:

\begin{figure}[h]
	\makebox[\textwidth] {
		\includegraphics[width=\textwidth]{llm_pipeline.png}
	}
	\captionof{figure}{Schematische voorstelling van een conversatie met een LLM chatbot \autocite{Ollama} }
\end{figure}

Hier valt op dat een LLM niet alleen de vraag van de gebruiker in acht neemt, maar ook relevante documenten opvraagt uit een Vector store. 

\subsubsection{Vectorstore: Chroma}
De eerste keuze die moet worden gemaakt is welke vectorstore te gebruiken. Veel bronnen op het web bespreken open-source databases zoals Weaviate, Opensearch, Qdrant en \textbf{Chroma}. 
Chroma biedt ondersteuning met LangChain en kan gemakkelijk lokaal gedraaid worden. 
Chroma is de keuze die door LangChain aangeraden wordt als lokaal alternatief voor een online instantie van Weaviate.  
Omdat we met gevoelige documenten werken, willen we Deltalex Chat graag volledig lokaal houden dus vandaar zullen we voor Chroma kiezen als onze lokale vectorstore. 

\subsubsection{Record store: PostgreSQL}
Een vectorstore is een grote ongestructureerde ruimte met alle berekende vectoren en documentdata. 
Het kan handig zijn om naast deze ruimte een externe database bij te houden die bijvoorbeeld transactionele data (welke documenten er in de vectorstore zijn gegaan en wanneer), 
gebruikersdata en andere gestructureerde data bijhoudt. Dit is in ons proof-of-concept niet van gigantisch belang dus houden we het hier op een relatief simpele open-source database  
die voor veel developers de eerste keuze is: \textbf{PostgreSQL}. 

\subsection{Centraal verwerkingssysteem}
Om alles samen te gieten en te beheren, hebben we een modulair systeem nodig dat we ofwel volledig zelf kunnen bouwen, of kunnen overnemen van een open-source framework. 
Eén van de populairste én best gedocumenteerde frameworks hiervoor is \textbf{LangChain}. 
De missie van Langchain is om de volledige levenscyclus van AI systemen makkelijker te maken om te beheren. 
Zo kan men zijn eigen systeem schrijven met modulaire bouwblokken die allemaal open-source zijn en dit integreren met third-party clients zoals AWS, Google, OpenAI, ... \\ 

De keuze voor Deltalex Chat zal dan ook naar LangChain gaan omdat het heel flexibel is. 
Allerbelangrijkst is het ook volledig lokaal draaibaar en open-source. 

\newpage
\subsection{Modelkeuze en beheer}
Een van de belangrijkste componenten van een digitale assistent is vanzelfsprekend artificiële intelligentie. 
Er wordt op twee plaatsen in de toepassing gebruikt gemaakt van AI: 

\begin{itemize}
	\item Tijdens het genereren van embeddings
	\item Tijdens het chatten en genereren van documenten
\end{itemize}

Voor deze twee doelen hebben we twee verschillende modellen nodig. 
Omdat alles lokaal zal draaien, hebben we een toepassing nodig op ons systeem om modellen te draaien en te beheren. 
Een toepassing die de laatste tijd heel erg populair aan het worden is in de open-source AI wereld, is \textbf{Ollama}. 
Ollama is een programma dat toestaat als gebruiker (via command line) modellen lokaal te downloaden, beheren en te draaien. 
De modellen kunnen via een interactieve Python prompt gebruikt worden of via een exposed API. 
Dit maakt het mogelijk om (via een community plugin) aangesproken te worden door LangChain. 

\subsubsection{Embedding model}
Het is niet mogelijk om documenten rechtstreeks in een vectorstore te plaatsen. 
Er moeten eerst van die documenten stukken tekst(chunks) gemaakt worden. 
Op basis van die chunks moeten er vectoren berekend worden, waarop de vectorstore kan checken of een stuk tekst relevant is met de query van de gebruiker of niet. 
Om deze te berekenen, wordt er gebruik gemaakt van een text-embedding model. 
Dit zal de tekstuele informatie overzetten naar dergelijke vectoren die compatibel zijn met Chroma. \\

Er zijn veel embedding modellen beschikbaar via Ollama, waarvan het populairste \textbf{nomic-embed-text}. 
Nomic heeft tot nu toe al rond de 350 duizend downloads van Ollama en is dus veruit het meest gedownloade. 
De keuze hiervoor is dan ook snel gemaakt. 

\subsubsection{Large language model}
Large Language Models zijn de laatste tijd gigantisch populair geworden door bv. ChatGPT van OpenAI.
Hun bijna menselijke interactie zorgt ervoor dat een gebruiker de indruk heeft dat hij aan het praten is met een menselijke assistent maar dan veel sneller en 'slimmer'.\\

Natuurlijk is niets minder waar.
Een Large Language Model is immers een soort transformer, die de input (of query) van een gebruiker 'transformeert' naar een antwoord door de volgende token (woord) te voorspellen.
Deze modellen kunnen gegeneraliseerd worden voor verschillende use cases, niet zoals hun voorgangers zoals een PLM (Pretrained Language Model).
Het verschil hier is vooral de gigantische dataset waarop LLM's getrained worden.
Dit maakt ze toepasbaar op een veel breder gebied zonder enige extra configuratie of aanpassing.

Er zijn de laatste tijd heel veel modellen (zowel open- als closed-source) op de markt gekomen. Dit maakt het moeilijk om te kiezen voor een large-language-model voor onze implementatie. 
Volgens populariteit bij het ollama modelregister zijn llama3, Gemma, Qwen en \textbf{Mistral} de bekendste. De eerste drie zijn van Meta, Google en Alibaba. \\

In deze proof-of-concept zal Mistral gebruikt worden. Het is natuurlijk altijd mogelijk om te testen tussen verschillende modellen om te zien wat de advocaat in kwestie verkiest. 

\section{Frontend}
Het is natuurlijk niet echt aantrekkelijk voor een advocaat om alles te doen via command line, aangezien dit voor hun niet echt intuïtief en gebruiksvriendelijk is. 
Daarom is het maken van een aantrekkelijke en gebruiksvriendelijke frontend een absolute must in onze POC. \\ 

Er zijn heel veel repositories online die al een vooraf geïmplementeerde chat-interface bevatten, die makkelijk via een NodeJS library kan gerund en getest worden. 
Deze oplossingen bieden een goede basis voor de ontwikkeling van een frontend, maar moeten aangepast worden om te conformeren aan de huisstijl van Deltalex en om te verzekeren dat ze alleen maar het absolute nodige weergeven. \\

De implementatie hier zal gebeuren met een NodeJS library genaamd \textbf{NextJS} en de communicatie zal verlopen via \textbf{FastAPI}.


In het volgende hoofdstuk zullen we verder ingaan op de implementatie van de frontend. 
We zullen met screenshots de werking verduidelijken en uitleggen. 

\chapter{Implementatie}
\label{ch:implementation}


\chapter{Resultaten en reflectie}
\label{ch:results_reflection}
\section{Behaalde resultaten}
In dit hoofdstuk volgt een objectieve beoordeling op de resultaten van dit onderzoek. \\

Het resultaat van deze bachelorproef zijn een paar zaken:

\begin{enumerate}
	\item Een literatuuronderzoek naar de huidige stand van zaken en een grondige beschrijving van het probleem
	\item Methodologie: Een opsomming van en onderzoek naar technologieën die enerzijds samen passen en anderzijds een positieve invloed hebben op ons probleem
	\item Een proof-of-concept van een digitale assistent
	\item De beoordeling van een advocaat van Deltalex en zodoende feedback over de impact op zijn workflow
\end{enumerate}

\subsection{Literatuurstudie}
De literatuurstudie is goed verlopen en de documenten gevonden zijn relevant en up-to-date.
De onderzoeksvraag is in subvragen verdeeld en individueel onderzocht en er is onderzoek gedaan naar mogelijke oplossingen voor ieder probleem.

\subsection{Methodologie}
De technologieën gevonden zijn open-source, up-to-date en lokaal inzetbaar.
Dit maakt ze ideaal voor de implementie van een proof-of-concept dat 100\% lokaal kan draaien.
Iedere technologie is besproken maar relatief oppervlakkig onderzocht.
Langdurig gebruik van de applicatie zal hun waarde uitwijzen.

\subsection{Proof-of-concept}
Het geleverde proof of concept is een fork van een veelgebruikt chatplatform gebouwd in het LangChain framework.
Dat maakt het een geteste oplossing die performant is out-of-the box als je in de cloud deployt.
De aanpassing van de app gebeurde met trial and error maar is uiteindelijk in orde gekomen.
Het embedding model heeft krachtig en snel zijn werk gedaan.
Het large-language model heeft soms wat moeite met spelling en kan soms wel eens hallucineren (foute informatie verzinnen),
maar de gegenereerde resultaten dienen louter ter inspiratie voor een advocaat om zijn werk op te baseren.

\section{Beoordeling van een advocaat uit Deltalex}


\subsection{Voorbeeldprompt en persoonlijke beoordeling}
\begin{lstlisting}
PROMPT: schrijf een ingebrekestelling voor 5000 euro aan aqualex
ANSWER:

	Ingebekeringverzoek

	Wij, [Naam en adres van de gemaakte bekering], vroegtelijk aangeduid als De Vermogende Party, op grond van artikel 475 ter, derde lid van het Gerechtelijk Wetboek, 
	hierbij verzoeken aan u, eerder bevoegd gezien bij de familie- rechtbank, een ingebrekestelling te doen van € 5.000,00 aan onze klantenrekening bij uw instelling, 
	onder nummer [Klantnummer], met vermelding dat deze ingebrekestelling gedaan wordt ter behandeling van de nalatenschap van onze overleden klant en volgens ons 
	kennisgevingen bedraagt het netto actief van de nalatenschap niet meer dan € 5.000,00.

	Bij aanvaarding van deze ingebrekestelling, zullen wij onze schulden tegenover uw instelling vervullen en gelijktijdig aan u het geld overmaken dat wordt gevorderd door de ingebrekestelling.

	Als de verwerper of verwerpers van de nalatenschap op eer verklaren dat volgens hun kennis het netto actief van de nalatenschap niet meer 
	bedraagt dan € 5.000,00, wordt deze ingebrekestelling uitgeoefend en vervult onze schulden aan uw instelling. 
	In geval dit geen geval is, zal ons verzoekschrift worden overweegd op de wijze omschreven in artikel 1231-4, tweede lid van het Gerechtelijk Wetboek.

	Wij danken u voor uw ondersteuning en verwachten u binnen een week te horen over de uitvoering van deze verzoekschrift.

	Voorzichtig en met zorg,
	[Naam en adres van de gemaakte bekering]
\end{lstlisting}

Minpunten:
\begin {itemize}
\item Veel spelfouten
\item Waarschijnlijk totaal verkeerde verwijzingen naar wetsartikels
\item Relatief kort voor dergelijk document
\end{itemize}

Pluspunten:
\begin{itemize}
	\item Professionele schrijfstijl
	\item Parameters worden schijnbaar niet in de wind geslagen en correct gebruikt
\end{itemize}


Citatie: TODO nog te ontvangen

\begin{lstlisting}
	Beste Laurent, ik schrijf u hierop betreffende jouw chatbot. 

	Ik heb jouw computer eens meegenomen naar kantoor, de toepassing een paar keer getest en had je beloofd eerstdaags schriftelijk mijn bedenkingen te bezorgen, dus bij deze. 
	
	De site is overzichtelijk en gebruiksvriendelijk. De chatbot is makkelijk in gebruik en maakt vrij snel een stuk tekst voor wat ik maar wil. 	

	Ik heb wel een paar bemerkingen. De bot kan wel incorrecte informatie genereren en ook spelfouten komen soms voor de dag. 

	In een geavanceerd stadium van ontwikkeling zou het ideaal zijn om deze fouten te corrigeren. 
	We zien wel hoe we hier verder mee gaan. 

	Uiteindelijk kan men zeggen dat het een interessant hulpmiddel kan zijn voor inspiratie en repetitieve taken zoals het starten van brieven. 

	Alvast bedankt om dit voor ons te realiseren!

	Tot snel, 
	Nicolas
\end{lstlisting}


\section{Reflectie}
Een reflectie verloopt heel goed aan de hand van auto-evaluatie d.m.v. vragen die men aan zichzelf stelt.
Hieronder volgen een paar vragen die kunnen ontspringen wanneer men denkt aan automatisatie van workflows en repetitieve taken.\\

\textbf{Welke taken zijn er nu makkelijk te automatiseren?}
De taken die hier in deze bachelorproef beschreven worden gaan over:
\begin{itemize}
	\item \textbf{Opzoekingswerk verrichten in documentendatabases}
	\item \textbf{Opzoekingswerk verrichten in publieke databases}
	\item \textbf{Opstellen van documenten}
\end{itemize}
Een digitale assistent kan een advocaat helpen met deze taken te versnellen door tekst te genereren ter inspiratie en met relevante informatie uit een database.
Het opstellen en gebruiken(versturen, indienen bij de rechtbank, ...) van deze documenten moet nog altijd gebeuren door de advocaat zelf.
Een digitale assistent dient niet ter vervanging van een advocaat, maar wel als een tool die hun workflow kan versnellen. \\

\textbf{Hoeveel productiviteitswinst kan een advocaat realiseren bij deze taken?}
Het gebruik van een digitale assistent zal opzoekingswerk versnellen en dienen ter inspiratie.
Om nog efficiënter te kunnen werken kan een advocaat technieken leren zoals sneltoetsen en tools gebruiken zoals browserextensies om hun workflow exponentieel te optimaliseren.
Hoeveel productiviteitswinst er dan is, hangt af van advocaat tot advocaat.
Sommigen zullen hun workflow bewaren zoals hij is en zijn er tevreden mee.
Anderen zullen altijd op zoek zijn om deze te versnellen en te optimaliseren en kunnen zich hier ook actief mee bezighouden.
Indien men de tijd wil vrijmaken om hier onderzoek naar te verrichten, zal er over de tijd een gestage stijging
zijn in de snelheid van hun workflow door het gebruik van technieken en digitale assistenten. \\

\textbf{Waarmee moet men rekening houden bij de ontwikkeling van een digitale assistent?}
Het belangrijkste aspect bij de ontwikkeling van een digitale assistent bij een advocatenkantoor is waarschijnlijk dataveiligheid.
Er mag absoluut, in geen enkele instantie twijfel ontstaan bij de gebruiker of bij de cliënt dat er vertrouwelijke data op het spel staat.
Een dergelijk voorval zou potentieel destructief zijn voor de klantenrelaties en het imago van het advocatenkantoor.
Andere belangrijke aspecten zijn gebruikersvriendelijkheid, snelheid van tokengeneratie en de kwaliteit van de gegenereerde antwoorden.\\

\textbf{Wat zijn de voor- en nadelen van een automatie via chatbots?}
De voordelen van chatbots zijn de snelheid van antwoorden, de bijna menselijke manier van interactie en de bron van inspiratie voor het opstellen van documenten.
Chatbots zoals ChatGPT zijn de laatste jaren aan een gigantische opmars bezig en hun functionaliteiten worden alleen maar beter dankzij de gigantisch hoge graad van evolutie in de technologische sector.
Nadelen van chatbots kunnen dingen zijn zoals hallucinatie(generatie van incorrecte antwoorden), grammaticale repetitie, ...
Het is altijd aangeraden om hun output te valideren i.p.v. er blind op te vertrouwen.

%%=============================================================================
%% Conclusie
%%=============================================================================

\chapter{Conclusie}%
\label{ch:conclusie}

% TODO: Trek een duidelijke conclusie, in de vorm van een antwoord op de
% onderzoeksvra(a)g(en). Wat was jouw bijdrage aan het onderzoeksdomein en
% hoe biedt dit meerwaarde aan het vakgebied/doelgroep? 
% Reflecteer kritisch over het resultaat. In Engelse teksten wordt deze sectie
% ``Discussion'' genoemd. Had je deze uitkomst verwacht? Zijn er zaken die nog
% niet duidelijk zijn?
% Heeft het onderzoek geleid tot nieuwe vragen die uitnodigen tot verder 
%onderzoek?

De uitkomst van dit onderzoek is een roadmap die informatie bevat over iedere component van de opbouw van een digitale assistent. 
Eerst was de opzet van dit onderzoek om een assistent te implementeren en te integreren met bestaande software. 

Natuurlijk werkt deze assistent met libraries en componenten die toch heel wat resources gebruiken om vloeiend te bouwen, draaien en testen. 
Dit in combinatie met de gevoeligheid van de cliëntdata is de beslissing gekomen om dit onderzoek puur hypothetisch uit te voeren. 
Denk over dit onderzoek als een stappenplan om een lezer bekend te maken met de basisconcepten van Natural Language Processing, Large Language Models, Retrieval Augmented Generation en meer. 

Veel stappen uit dit onderzoek worden minder breed besproken gezien het hedendaagse aanbod van digitale tools zoals JavaScript libraries, 
frameworks die al heel goeie implementaties voor digitale assistenten leveren en de extraordinaire snelle aard van verandering van het huidig technologisch spectrum. 

Er is in dit onderzoek geprobeerd om zo veel mogelijk technologiespecifiek te werken. 
Het kan goed zijn dat er binnenkort betere technologieën uitgebracht worden die veel beter presteren dan degene die hier gebruikt worden. 
De technologieën gebruikt in dit onderzoek zijn op het moment van onderzoek de meest geschikte en performante op de markt, waarvan de meeste open-source. 

In conclusie hoop ik dat een lezer van dit onderzoek enerzijds bijleert over digitale assistenten en hoe ze werken onder de motorkap. 
Anderzijds dat hij het ook kan gebruiken als stappenplan tijdens het implementeren van zijn eigen assistent, iets waar de student die er nu aan werkt ook zeker gebruik van zal maken. 


% Voeg hier je eigen hoofdstukken toe die de ``corpus'' van je bachelorproef
% vormen. De structuur en titels hangen af van je eigen onderzoek. Je kan bv.
% elke fase in je onderzoek in een apart hoofdstuk bespreken.

\chapter{Documentdatabase}
Er zijn verschillende stappen nodig om een grote hoeveelheid documenten te gieten in een database.
Deze worden hieronder stap voor stap behandeld.

\section{Inname van documenten}
Nadat de verzameling van documenten gebeurd is, zijn deze natuurlijk allemaal in pdf, docx of dergelijk documentformaat.
Het is natuurlijk de bedoeling dat al deze documenten naar tekst te converteren zodat een taalmodel er efficiënt mee overweg kan.
Om text te extraheren uit documenten kan men verschillende tools gebruiken.
Na wat research online zijn er heel veel verschillende alternatieven, maar de meeste zijn betalend en al gegoten in programma's die gebruiksklaar zijn.
Om over al die tools te beschikken en om via een uniforme manier tekst te extraheren, is er een heel goeie tool in Java, Apache Tika.

\subsection{Apache Tika}
Tika is een framework gemaakt door Apache, dit specialiseert op het extraheren van gestructureerde tekstdata en metadata uit verschillende documentdatatypes.
De sterkte van Tika is dat het eigenlijk een wrapper is rond verschillende bibliotheken en ze allemaal samenbrengt in één tool.
Al deze verschillende bestandstypes kunnen door één enkele interface worden verwerkt.
Dit maakt Tika een heel goeie tool voor zoekmachineindexering, tekstanalyse en vertaling.
Deze tool zou een goede kandidaat zijn om in eerste instantie alle data van alle bestanden te converteren naar bijvoorbeeld pure tekst of HTML.

Met een tool zoals JavaMail kan men een inbox uitlezen en recursief deze mails individueel behandelen, dit maakt het makkelijk om deze content in onze database in te brengen.
Het voordeel van een totaal mailbestand in onze database te plaatsen,
is dat een assistent dan efficiënter is in het opstellen van communicatie die zal matchen met de schrijfstijl van de advocaat waar de mailbox aan toebehoort.

\section{Opzetten van database}
Nadat er documenten omgezet zijn in tekst, moeten deze in een database terechtkomen.
Dit is allemaal tekstdata, dus het is vanzelfsprekend dat we kiezen voor een database engine die goed is met veel ongestructureerde data.
Twee voorbeelden hiervan zijn MongoDB en OpenSearch:

\subsection{MongoDB}
MongoDB is een open-source NoSQL (Not-only-SQL) documentgeörienteerde database engine die gedraaid kan worden op fully-managed cloud service of self-hosted.
MongoDB kan op heel veel verschillende besturingssystemen draaien (vooral GNU/Linux servers) en het hoofddoel van hun cloud services is MongoDB Atlas, dat wereldwijde,
ongezien snelle distributie en mobiliteit biedt over de bekendste cloudproviders, zoals Google Cloud, Amazon Web Services en Microsoft Azure.

De conclusie is dat MongoDB heel goed is voor heel veel ongestructureerde data en snelheid, maar als een zoekmachine zijn er betere alternatieven.

\subsection{OpenSearch}
OpenSearch is ook een open-source NoSQL database engine die gefocust is op het zoeken van data.
Deze is geforked(gebouwd bovenop een open-source repository) op Elasticsearch en gebaseerd op Apache Lucene(Open-source zoeklibrary).

Het voordeel van OpenSearch is dat het native op Java draait.
Dit biedt mogelijkheden om ons innamesysteem van documenten (dat ook op Java zal draaien) makkelijk te integreren met ons databasesysteem.
Dit zijn bouwstenen naar een microservice-gebaseerd systeem dat (elke service geïsoleerd) samen opgezet kan worden.

Dit in combinatie met het gegeven dat OpenSearch gefocust is op het zoeken van data, laat de beslissing naar deze kant leunen als uitverkoren databasesysteem.

\section{Metadatasysteem}
Het doel van een metadatasysteem is om over elk opgeslagen document bepaalde informatie op te slaan.

Voorbeelden van deze metadata kunnen zijn:
\begin{itemize}
	\item \textbf{Bestandstype}
	\item \textbf{Auteur}
	\item \textbf{Datum van opstelling}
	\item \textbf{Contextuele data:} Categorie, onderwerp van een email, afzender, \dots
\end{itemize}

De opslagplaats van deze data wordt opgeslagen naast onze documentdata.
Samen laat dit toe om efficient doorzocht te worden.

\section{Indexering}
Het creëren van indexen is een van de belangrijkste zaken in het bouwen van een efficiënt databasesysteem.
De snelheid van de zoekfuncties (dus in extensie de snelheid van de implementatie in het algemeen) wordt rechtstreeks beïnvloed door het gebruik van al dan niet efficiënte indexen.

Een snelle uitleg van indexeren in databasesystemen hebben we te danken aan learnsql.com:

\begin{displayquote}
	"When the database searches in an index, the first step is to find the key value in the index.
	Every key value is stored with a pointer to the record in the table associated with this key value.
	Then, when the key value is found, the database follows that pointer and directly reads the record(s) from the table."
	\autocite{learnSQL}
\end{displayquote}

Als deze stappen gevolgd worden, zal alle data efficient opgeslagen worden in ons databasesysteem. 

Zodoende is de data klaar voor de volgende stap: chunking van de data. 

\chapter{Context-aware chunking van verzamelde data}
Nu alle data in een juist formaat en met metadata in de database steekt, is het tijd voor het volgende ordeningsprincipe: chunking.
Chunking is het principe van extensieve ongestructureerde data op te delen in kleinere, logische stukken, oftewel chunks.
Dit proces kan verdeeld worden in verschillende stappen.

\begin{enumerate}
	\item \textbf{Voorbereiden en reinigen van data:} Bekijk alle data en verwijder onnodige opmaak, titels en punctuatie. Deze stap is optioneel maar het
	      kan altijd helpen om de verzamelde data eens te overlopen en zien of er geen grote gaten in zitten.
	\item \textbf{Tokenization van de tekst:} De data kan nu onderverdeeld worden in tokens, oftewel individuele woorden.
	      Dit kan bereikt worden met algoritmes zoals SpaCy of NLTK,
	      dit zijn Natural Language Processing algoritmes.
	\item \textbf{Het detecteren van zinnen:} Nu de data verdeeld is in tokens, is het tijd om met de hulp van sentence boundary algoritmes de zinnen te
	      detecteren die coherente zinnen vormen. Er zijn ook veel NLP bibliotheken die deze algoritmes implementeren.
	\item \textbf{Groeperen van zinnen in chunks:} Nu de zinnen verdeeld zijn, is het tijd om deze te verdelen in groepen, oftewel chunks.
	      De manier waarop dit kan gebeuren kan via een woordlimiet per chunk, een specifieke topic per chunk of een semantische/ syntactische gelijkenis van zinnen.
\end{enumerate}

Nu deze chunks gegroepeerd zijn, vervangen we de data door de chunks in de database en slaan we ze op samen met de metadata van de documenten.
Deze chunks kunnen voorgesteld worden als vectoren en kunnen samen geïmplementeerd worden in een vector database.
Dit type database slaat vectoren op op basis van hun similariteit met andere vectors en deze databases gebruiken veelal het Approximate Nearest Neighbour algorithm.

Een citaat van rockset.com verduidelijkt dit principe:

\begin{displayquote}
	With the evolution of machine learning, neural networks and large language models, organizations can easily transform unstructured data into embeddings, commonly represented as vectors.
	Vector search operates across these vectors to identify patterns and quantify similarities between components of the underlying unstructured data. 
	\autocite{Rockset}
\end{displayquote}

Nu de data verdeeld is in chunks, kunnen we verdergaan naar de volgende stap, het integreren van Ollama met onze database.

\chapter{Integreren van Ollama met documentdatabase}

Nu we een geordende documentendatabase hebben, is het tijd om een LLM (Large Language Model) te configureren zodat het kan werken via RAG en zich zal voordoen als een digitale assistent.

Eerst hebben we een retrieval systeem nodig.
Dit is essentieel om queries naar onze database te sturen en chunks op te halen en terug te geven.
Deze documenten worden dan samengevoegd met de query van de gebruiker en gebruikt als parameters voor een Large Language Model. µ

\section{Retrievalsysteem}
\subsection{Werking}
Een retrievalsysteem haalt (om het simpel uit te leggen) data op uit een set data die 'overeenstemt' met de query.
Maar wat is een gelijkenis? Hoe gelijk?
Hoe gaan we documenten vergelijken? Laten we het eens op een rijtje zetten.

\subsection{Retrievalmodellen}
Retrievalmodellen zijn wiskundige technieken die toegepast worden in o.a. zoekmachines en deze zijn gemaakt om relevante documenten op te snorren die relevant zijn met de query van de gebruiker.
We zullen hier drie modellen bespreken die veel gebruikt worden.
In realiteit worden er echter meer libraries gebruikt (gevonden op b.v. het LangChain of Huggingface platform, deze zijn forges die toegang bieden tot allerlei implementaties van LLM tot videogeneratie),
waar de keuze van retrievalmodel voor ons al gemaakt is of tijdens runtime dynamisch gekozen wordt.

\begin{enumerate}
	\item \textbf{Booleaanse retrieval:} dit is een van de meest simpele modellen.
	      Het gebruikt booleaanse (OR, AND, NOT) operatoren om een match te evalueren.
	      Dit kan heel efficiënt en snel zijn, maar het heeft een paar beperkingen.
	      Enerzijds heeft het geen manier om gevonden documenten te ordenen op relevantie
	      en anderzijds kan het heel veel irrelevante informatie teruggeven als een query te breed is gesteld.

	\item \textbf{Vectorruimtemodellen:} Deze zijn wat complexer.
	      Ze gebruiken vectoren, dit zijn documenten en queries die voorgesteld worden in een high-dimensional space.
	      Iedere dimensie is een soort index en de waarde ervan representeert het 'gewicht' of belangrijkheid van de index.
	      Omdat deze vectors 'gewogen' zijn, kan een systeem gelijkheid gaan berekenen en de dichtste documenten met de query vector teruggeven.
	      Zodoende kan er ook berekend worden hoe relevant een resultaat is met de query.
	      Het voordeel is hier dat onze database al gerepresenteerd is als vectoren en dus efficiënt kan doorzocht worden door dergelijke modellen.

	\item \textbf{Probalistische modellen: } Deze modellen gebruiken statistiek om de kans in te schatten dat een document relevant is met de query of niet.
	      Deze zijn gemaakt om complexere queries te behandelen en kunnen om met een 'onzeker' resultaat, waar een booleaans model zal falen.
\end{enumerate}

Deze zijn drie voorbeelden van modellen. Merk op dat het eerste het simpelste is en ook bespaart op rekenkracht,
waar de laatste twee modellen exponentieel meer resources zullen gebruiken gebaseerd op de grootte van de database.

Zoals eerder vernoemd kan het een snellere manier zijn om libraries te gebruiken (die al getest zijn op performantie) in plaats van een volledig eigen systeem te schrijven.

\section{Large Language Model}
\subsection{Introductie}
Large Language Models zijn de laatste tijd gigantisch populair geworden door bv. ChatGPT van OpenAI.
Hun bijna menselijke interactie zorgt ervoor dat een gebruiker de indruk heeft dat hij aan het praten is met een menselijke assistent maar dan veel sneller en 'slimmer'.

Natuurlijk is niets minder waar.
Een Large Language Model is immers een soort transformer, die de input (of query) van een gebruiker 'transformeert' naar een antwoord door de volgende token (woord) te voorspellen.
Deze modellen kunnen gegeneraliseerd worden voor verschillende use cases, niet zoals hun voorgangers zoals een PLM (Pretrained Language Model).
Het verschil hier is vooral de gigantische dataset waarop LLM's getrained worden.
Dit maakt ze toepasbaar op een veel breder gebied zonder enige extra configuratie of aanpassing.
Er zijn heel veel papers geschreven over hun evolutie en werking, hieronder een citatie uit een ervan:

\begin{displayquote}
	"Recently, significant breakthroughs have been witnessed in language models, primarily attributed to transformers, increased computational capabilities, and the availability of large-scale training data.
	These developments have brought about a revolutionary transformation by enabling the creation of LLMs that can approximate human-level performance on various tasks.
	Large Language Models (LLMs) have emerged as cutting-edge artificial intelligence systems that can process and generate text with coherent communication, and generalize to multiple tasks."
	\begin{figure}[h]
		\makebox[\textwidth]{
			\includegraphics[width=\textwidth]{LLMs_Overview.png}
		}
		\centering
	\end{figure}
	\autocite{ACOoLLM}


\end{displayquote}

Large Language Models zijn heel complexe en extensieve implementaties, hierboven bij het citaat staat een diagram dat in vijf takken uitlegt hoe ze oppervlakkig in elkaar zitten.

\subsection{Keuze van een model}
Het kiezen van een model is iets dat van verschillende factoren afhangt:

\begin{itemize}
	\item \textbf{Modelgrootte en capabiliteit:} Grotere modellen geven meestal meer functionaliteiten en snellere werking maar dit ten koste van computerresources.
	\item \textbf{Datering van trainingsdata:} Oudere modellen die voorgetraind zijn op bepaalde data kunnen niet meer up-to-date zijn kunnen zo inaccurate antwoorden genereren.
	\item \textbf{Toegankelijkheid en kost:} Sommige modellen zijn betalend, terwijl sommige open-source modellen gratis te gebruiken zijn.
\end{itemize}

De keuze van een model hangt ook af van tijd; alles evolueert heel snel dus het is aangeraden om van tijd tot tijd te evalueren of ons model niet kan vervangen worden door iets nieuwer en capabeler.

Er zijn heel veel online providers van LLM apps zoals ChatGPT, Bard, Grok enz. 
Het probleem met cloudproviders is dat je met vertrouwelijke cliëntdata zit, deze wordt idealiter niet over het internet gestuurd naar een (gratis) LLM API. 

De keuze is dus, zoals eerder vermeld in deze paper, voor een lokaal model.
Deze zijn kneedbaar in implementatie en te verbinden met een vector store zoals degene die wij gebruiken. 


Nu neigt de keuze naar llama-3 van Meta: Een van de meest capabele modellen die openbaar verkrijgbaar zijn op dit moment.
\newpage
\section{Samenbrengen}
Nu de technologie van onze backend gekozen is, is het tijd om het samen te brengen in een entiteit die een input ontvangt en op basis van die input en de documentdatabase een output verschaft.

Een artikel van Jacob Lee op de blog van ollama.com wijst ons de weg op een straightforward manier.

\begin{displayquote}
	The general idea here is to take the user’s input question, search our prepared vectorstore for document chunks most semantically similar to the query,
	and use the retrieved chunks plus the original question to guide the LLM to a final answer based on our input data.

	There’s an additional step required for followup questions, which may contain pronouns or other references to prior chat history.
	Because vectorstores perform retrieval by semantic similarity, these references can throw off retrieval.
	Therefore, we add an additional dereferencing step that rephrases the initial step into a “standalone” question before using that question to search our vectorstore. 

	\begin{figure}[h]
		\makebox[\textwidth] {
			\includegraphics[width=\textwidth]{llm_pipeline.png}
		}
	\end{figure}
	\autocite{Ollama}
\end{displayquote}

Bovenstaande pipeline is een heel goeie richtlijn om onze implementatie op te baseren. Nu hebben we een interface nodig om queries in op te stellen als gebruiker en de outputs te kunnen lezen in een chat-achtige interface. 

\chapter{Implementeren van een chatfrontend}
Nu we de backend hebben voor onze digitale assistent, is het tijd om er een aantrekkelijke, overzichtelijke en snelle frontend interface aan te verbinden. 
Er zijn heel veel frontend libraries die via NodeJS, Deno, Bun en andere JavaScript frameworks kunnen gedeployed worden in een paar seconden. 

In dit voorbeeld zullen we een template gebruiken van NextJS en aanpassen naar onze wens. 
Het voordeel hier is dat we geen tijd moeten spenderen aan het bouwen van een interface omdat we al een mooie, overzichtelijke interface hebben. Deze ziet er uit als onderstaande screenshot:

\begin{figure}[h]
	\makebox[\textwidth]{
		\includegraphics[width=\textwidth]{chat_frontend.png}
	}
	\captionof{figure}{Screenshot van NextJS template chat frontend}
\end{figure}
\newpage

Deze chatbot komt standaard met OpenAI API ingebouwd. 
Er zal dus wat sleutelwerk nodig zijn om deze te linken met onze Ollama interne API, maar de open-source strategie van deze chatapp maakt ons dit gelukkig vrij gemakkelijk. 
NodeJS maakt het ons ook makkelijk om op een lokale machine bij Deltalex te deployen. 
Op deze manier kunnen we het volledige systeem isoleren en lokaal houden, om het zo veilig mogelijk te houden. 

Het nadeel van deze lokale aanpak is dat er een sterke grafische kaart in de lokale machine moet zitten om modellen performant te kunnen draaien voor meerdere gebruikers op hetzelfde tijdstip. 
Daarom is deze paper slechts een roadmap die de lezer tot een volledige, werkelijke implementatie leidt. 

\chapter{Opzetten van lokale omgeving en componenten verbinden}
Nu het tijd is om alles samen te brengen, volgen we de volgende stappen om alles lokaal te draaien:

\section{Frontend voorbereiden}
Onze Next.js frontend stellen we in om HTTP requests en responses te faciliteren van en naar ons Langchain systeem. 
Dit zal vrij eenvoudig verlopen omdat veel implementatie al voor ons gedaan is in het template project. 

\section{RAG systeem voorbereiden}
Langchain is een framework dat naast het controleren van onze modellen en vector stores ook een REST API host, die gecalled kan worden door onze chat-frontend. 
Op deze manier houden we onze frontend en backend gescheiden van elkaar en verbinden we ze enkel over HTTP netwerkverkeer. 
We kunnen dus op de machine van iedere advocaat een frontend installeren die veilig communiceert met één interne backend. 

Nu deze twee met elkaar kunnen communiceren, zal het systeem intensief getest moeten worden op zwaar verkeer, incorrecte input, foute antwoorden van ons LLM, enz. 


\chapter{Testing}

Testing van onze RAG chat tool is een proces dat oppervlakkig verdeeld kan worden in drie hoofdstukken. 

\section{Testen van het systeem}
Een eerste stap is \textbf{functioneel testen}, dit is de meest basic vorm van testen en kijkt of alle functies van het systeem al dan niet werken. 
Denk maar aan gebruikersinput, query output, of het retrievalsysteem wel relevante chunks ophaalt, ...
Daarna komt \textbf{performance testing}. Dit evalueert of het systeem onder verschillende loads naar behoren presteert. 
Dit type testen wordt gebruikt om mogelijke bottlenecks te elimineren en een performant systeem te bekomen. 
Als laatst hebben we \textbf{user experience testing}. Hier verwerken we feedback van de gebruikers over de bruikbaarheid van het systeem. 
Dit houdt zaken in zoals lay-out, snelheid, gebruikersvriendelijkheid, ... 

\section{Optimalisatie van het systeem}
Dit hoofdstuk kan eigenlijk oneindig lang herhaald worden. 
Er zijn altijd dingen die betere kunnen en we kunnen ze identificeren door volgende technieken toe te passen:

\begin{itemize}
	\item \textbf{Iteratieve optimalisatie:} Het constant verwerken van de feedback van advocaten en observeren van de evolutie van data kan de antwoorden van onze assistent accurater en behulpzamer maken. 
	\item \textbf{Veranderen van het model:} Als je een beetje de actualiteit rond LLMs volgt, kan het zijn dat er plots eentje op de proppen komt die llama3 zal outperformen. 
		Het is aangeraden om hierover up-to-date te blijven. 
	\item \textbf{Robuustheid:} Observeren van logbestanden van Langchain kan ons naar memory leaks, fouten in chunking of embeddings en andere leiden. 
		Robuustheid kan ook verbeterd worden door geoptimaliseerde foutbehandeling. 
\end{itemize}

Deze lijst kan ook toegepast worden als we willen schalen naar grotere datasets/ een groter aantal gebruikers. 

\section{Monitoring en feedback}
Zoals in het vorige hoofdstuk al aangewezen, is het van elementair belang dat we het systeem constant in de gaten houden. 
Dit kan door rechtstreeks logs uit te lezen, of deze te sturen naar een logging- en visualisatieframework zoals Grafana.
Data is alles en op basis van die data kunnen we blijven optimaliseren en verbeteren. 




%---------- Bijlagen -----------------------------------------------------------

\appendix

\chapter{Onderzoeksvoorstel}

Het onderwerp van deze bachelorproef is gebaseerd op een onderzoeksvoorstel dat vooraf werd beoordeeld door de promotor. Dat voorstel is opgenomen in deze bijlage.

%% TODO: 
%\section*{Samenvatting}

% Kopieer en plak hier de samenvatting (abstract) van je onderzoeksvoorstel.

% Verwijzing naar het bestand met de inhoud van het onderzoeksvoorstel
%---------- Inleiding ---------------------------------------------------------

\section{Introductie}%
\label{sec:introductie}

Mijn bachelorproef spitst zich toe op het optimaliseren van de administratieve workload bij advocatenkantoor Deltalex door het invoeren van geautomatiseerde processen, alsook het
ontwikkelen van een centrale webinterface om opzoekwerk te verrichten en ook de sjablonen van de meestgebruikte documenten bij te houden en te bewerken.

Dit onderzoek vindt zijn oorsprong in de vaststelling van veel tijdrovende en repetitieve taken die de efficiëntie van dit kantoor kunnen belemmeren,
zoals het handmatig opzoeken van cliëntgegevens in openbare databases en het bijwerken van veelgebruikte documenttemplates.

De onderzoeksvraag van mijn bachelorproef klinkt als volgt: "Hoe kunnen geautomatiseerde processen, in combinatie met een webinterface voor beheer, de administratieve
last in advocatenkantoren verminderen en de efficiëntie bevorderen?"

Het beoogde eindresultaat omvat niet alleen een werkend prototype van de geautomatiseerde toepassing en de bijbehorende webinterface,
maar ook een grondig rapport met aanbevelingen voor implementatie in advocatenkantoren.

Het succes van deze bachelorproef zal worden beoordeeld aan de hand van daadwerkelijke verbeteringen in efficiëntie en
tijdsbesparing binnen het advocatenkantoor Deltalex.

%---------- Stand van zaken ---------------------------------------------------

\section{State-of-the-art}%
\label{sec:state-of-the-art}

Hier beschrijf je de \emph{state-of-the-art} rondom je gekozen onderzoeksdomein, d.w.z.\ een inleidende, doorlopende tekst over het onderzoeksdomein van je bachelorproef. Je steunt daarbij heel sterk op de professionele \emph{vakliteratuur}, en niet zozeer op populariserende teksten voor een breed publiek. Wat is de huidige stand van zaken in dit domein, en wat zijn nog eventuele open vragen (die misschien de aanleiding waren tot je onderzoeksvraag!)?

Je mag de titel van deze sectie ook aanpassen (literatuurstudie, stand van zaken, enz.). Zijn er al gelijkaardige onderzoeken gevoerd? Wat concluderen ze? Wat is het verschil met jouw onderzoek?

Verwijs bij elke introductie van een term of bewering over het domein naar de vakliteratuur, bijvoorbeeld~\autocite{Hykes2013}! Denk zeker goed na welke werken je refereert en waarom.

Draag zorg voor correcte literatuurverwijzingen! Een bronvermelding hoort thuis \emph{binnen} de zin waar je je op die bron baseert, dus niet er buiten! Maak meteen een verwijzing als je gebruik maakt van een bron. Doe dit dus \emph{niet} aan het einde van een lange paragraaf. Baseer nooit teveel aansluitende tekst op eenzelfde bron.

Als je informatie over bronnen verzamelt in JabRef, zorg er dan voor dat alle nodige info aanwezig is om de bron terug te vinden (zoals uitvoerig besproken in de lessen Research Methods).

% Voor literatuurverwijzingen zijn er twee belangrijke commando's:
% \autocite{KEY} => (Auteur, jaartal) Gebruik dit als de naam van de auteur
%   geen onderdeel is van de zin.
% \textcite{KEY} => Auteur (jaartal)  Gebruik dit als de auteursnaam wel een
%   functie heeft in de zin (bv. ``Uit onderzoek door Doll & Hill (1954) bleek
%   ...'')

Je mag deze sectie nog verder onderverdelen in subsecties als dit de structuur van de tekst kan verduidelijken.

%---------- Methodologie ------------------------------------------------------
\section{Methodologie}%
\label{sec:methodologie}

Hier beschrijf je hoe je van plan bent het onderzoek te voeren. Welke onderzoekstechniek ga je toepassen om elk van je onderzoeksvragen te beantwoorden? Gebruik je hiervoor literatuurstudie, interviews met belanghebbenden (bv.~voor requirements-analyse), experimenten, simulaties, vergelijkende studie, risico-analyse, PoC, \ldots?

Valt je onderwerp onder één van de typische soorten bachelorproeven die besproken zijn in de lessen Research Methods (bv.\ vergelijkende studie of risico-analyse)? Zorg er dan ook voor dat we duidelijk de verschillende stappen terug vinden die we verwachten in dit soort onderzoek!

Vermijd onderzoekstechnieken die geen objectieve, meetbare resultaten kunnen opleveren. Enquêtes, bijvoorbeeld, zijn voor een bachelorproef informatica meestal \textbf{niet geschikt}. De antwoorden zijn eerder meningen dan feiten en in de praktijk blijkt het ook bijzonder moeilijk om voldoende respondenten te vinden. Studenten die een enquête willen voeren, hebben meestal ook geen goede definitie van de populatie, waardoor ook niet kan aangetoond worden dat eventuele resultaten representatief zijn.

Uit dit onderdeel moet duidelijk naar voor komen dat je bachelorproef ook technisch voldoen\-de diepgang zal bevatten. Het zou niet kloppen als een bachelorproef informatica ook door bv.\ een student marketing zou kunnen uitgevoerd worden.

Je beschrijft ook al welke tools (hardware, software, diensten, \ldots) je denkt hiervoor te gebruiken of te ontwikkelen.

Probeer ook een tijdschatting te maken. Hoe lang zal je met elke fase van je onderzoek bezig zijn en wat zijn de concrete \emph{deliverables} in elke fase?

%---------- Verwachte resultaten ----------------------------------------------
\section{Verwacht resultaat, conclusie}%
\label{sec:verwachte_resultaten}

Hier beschrijf je welke resultaten je verwacht. Als je metingen en simulaties uitvoert, kan je hier al mock-ups maken van de grafieken samen met de verwachte conclusies. Benoem zeker al je assen en de onderdelen van de grafiek die je gaat gebruiken. Dit zorgt ervoor dat je concreet weet welk soort data je moet verzamelen en hoe je die moet meten.

Wat heeft de doelgroep van je onderzoek aan het resultaat? Op welke manier zorgt jouw bachelorproef voor een meerwaarde?

Hier beschrijf je wat je verwacht uit je onderzoek, met de motivatie waarom. Het is \textbf{niet} erg indien uit je onderzoek andere resultaten en conclusies vloeien dan dat je hier beschrijft: het is dan juist interessant om te onderzoeken waarom jouw hypothesen niet overeenkomen met de resultaten.



%%---------- Andere bijlagen --------------------------------------------------
% TODO: Voeg hier eventuele andere bijlagen toe. Bv. als je deze BP voor de
% tweede keer indient, een overzicht van de verbeteringen t.o.v. het origineel.
%\input{...}

%%---------- Backmatter, referentielijst ---------------------------------------

\backmatter{}

\setlength\bibitemsep{2pt} %% Add Some space between the bibliograpy entries
\printbibliography[heading=bibintoc]

\end{document}
