\chapter{Opzetten van documentendatabase}
Er zijn verschillende stappen nodig om een grote hoeveelheid documenten te gieten in een database.
Deze worden hieronder stap voor stap behandeld.

\section{Inname van documenten}
Nadat de verzameling van documenten gebeurd is, zijn deze natuurlijk allemaal in pdf, docx of dergelijk documentformaat.
Het is natuurlijk de bedoeling dat al deze documenten naar tekst te converteren zodat een taalmodel er efficiënt mee overweg kan.
Om text te extraheren uit documenten kan men verschillende tools gebruiken.
Na wat research online zijn er heel veel verschillende alternatieven, maar de meeste zijn betalend en al gegoten in programma's die gebruiksklaar zijn.
Om over al die tools te beschikken en om via een uniforme manier tekst te extraheren, is er een heel goeie tool in Java, Apache Tika.

\subsection{Apache Tika}
Tika is een framework gemaakt door Apache, dit specialiseert op het extraheren van gestructureerde tekstdata en metadata uit verschillende documentdatatypes.
De sterkte van Tika is dat het eigenlijk een wrapper is rond verschillende bibliotheken en ze allemaal samenbrengt in één tool.
Al deze verschillende bestandstypes kunnen door één enkele interface worden verwerkt.
Dit maakt Tika een heel goeie tool voor zoekmachineindexering, tekstanalyse en vertaling.
Deze tool zou een goede kandidaat zijn om in eerste instantie alle data van alle bestanden te converteren naar bijvoorbeeld pure tekst of HTML.

Met een tool zoals JavaMail kan men een inbox uitlezen en recursief deze mails individueel behandelen, dit maakt het makkelijk om deze content in onze database in te brengen.
Het voordeel van een totaal mailbestand in onze database te plaatsen,
is dat een assistent dan efficiënter is in het opstellen van communicatie die zal matchen met de schrijfstijl van de advocaat waar de mailbox aan toebehoort.

\section{Opzetten van database}
Nadat er documenten omgezet zijn in tekst, moeten deze in een database terechtkomen.
Dit is allemaal tekstdata, dus het is vanzelfsprekend dat we kiezen voor een database engine die goed is met veel ongestructureerde data.
Twee voorbeelden hiervan zijn MongoDB en OpenSearch:

\subsection{MongoDB}
MongoDB is een open-source NoSQL (Not-only-SQL) documentgeörienteerde database engine die gedraaid kan worden op fully-managed cloud service of self-hosted.
MongoDB kan op heel veel verschillende besturingssystemen draaien (vooral GNU/Linux servers) en het hoofddoel van hun cloud services is MongoDB Atlas, dat wereldwijde,
ongezien snelle distributie en mobiliteit biedt over de bekendste cloudproviders, zoals Google Cloud, Amazon Web Services en Microsoft Azure.

De conclusie is dat MongoDB heel goed is voor heel veel ongestructureerde data en snelheid, maar als een zoekmachine zijn er betere alternatieven.

\subsection{OpenSearch}
OpenSearch is ook een open-source NoSQL database engine die gefocust is op het zoeken van data.
Deze is geforked(gebouwd bovenop een open-source repository) op Elasticsearch en gebaseerd op Apache Lucene(Open-source zoeklibrary).

Het voordeel van OpenSearch is dat het native op Java draait.
Dit biedt mogelijkheden om ons innamesysteem van documenten (dat ook op Java zal draaien) makkelijk te integreren met ons databasesysteem.
Dit zijn bouwstenen naar een microservice-gebaseerd systeem dat (elke service geïsoleerd) samen opgezet kan worden.

Dit in combinatie met het gegeven dat OpenSearch gefocust is op het zoeken van data, laat de beslissing naar deze kant leunen als uitverkoren databasesysteem.

\section{Metadatasysteem}
Het doel van een metadatasysteem is om over elk opgeslagen document bepaalde informatie op te slaan.

Voorbeelden van deze metadata kunnen zijn:
\begin{itemize}
	\item \textbf{Bestandstype}
	\item \textbf{Auteur}
	\item \textbf{Datum van opstelling}
	\item \textbf{Contextuele data:} Categorie, onderwerp van een email, afzender, \dots
\end{itemize}

De opslagplaats van deze data wordt opgeslagen naast onze documentdata.
Samen laat dit toe om efficient doorzocht te worden.

\section{Indexering}
Het creëren van indexen is een van de belangrijkste zaken in het bouwen van een efficiënt databasesysteem.
De snelheid van de zoekfuncties (dus in extensie de snelheid van de implementatie in het algemeen) wordt rechtstreeks beïnvloed door het gebruik van al dan niet efficiënte indexen.

Een snelle uitleg van indexeren in databasesystemen hebben we te danken aan learnsql.com:

\begin{displayquote}
	"When the database searches in an index, the first step is to find the key value in the index.
	Every key value is stored with a pointer to the record in the table associated with this key value.
	Then, when the key value is found, the database follows that pointer and directly reads the record(s) from the table."
	\autocite{learnSQL}
\end{displayquote}

Als deze stappen gevolgd worden, zal alle data efficient opgeslagen worden in ons databasesysteem. 

Zodoende is de data klaar voor de volgende stap: chunking van de data. 
