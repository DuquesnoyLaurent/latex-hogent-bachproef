%%=============================================================================
%% Samenvatting
%%=============================================================================

% TODO: De "abstract" of samenvatting is een kernachtige (~ 1 blz. voor een
% thesis) synthese van het document.
%
% Een goede abstract biedt een kernachtig antwoord op volgende vragen:
%
% 1. Waarover gaat de bachelorproef?
% 2. Waarom heb je er over geschreven?
% 3. Hoe heb je het onderzoek uitgevoerd?
% 4. Wat waren de resultaten? Wat blijkt uit je onderzoek?
% 5. Wat betekenen je resultaten? Wat is de relevantie voor het werkveld?
%
% Daarom bestaat een abstract uit volgende componenten:
%
% - inleiding + kaderen thema
% - probleemstelling
% - (centrale) onderzoeksvraag
% - onderzoeksdoelstelling
% - methodologie
% - resultaten (beperk tot de belangrijkste, relevant voor de onderzoeksvraag)
% - conclusies, aanbevelingen, beperkingen
%
% LET OP! Een samenvatting is GEEN voorwoord!

%%---------- Nederlandse samenvatting -----------------------------------------
%
% TODO: Als je je bachelorproef in het Engels schrijft, moet je eerst een
% Nederlandse samenvatting invoegen. Haal daarvoor onderstaande code uit
% commentaar.
% Wie zijn bachelorproef in het Nederlands schrijft, kan dit negeren, de inhoud
% wordt niet in het document ingevoegd.

\IfLanguageName{english}{%
\selectlanguage{dutch}
\chapter*{Samenvatting}
\selectlanguage{english}
}{}

%%---------- Samenvatting -----------------------------------------------------
% De samenvatting in de hoofdtaal van het document

\chapter*{\IfLanguageName{dutch}{Samenvatting}{Abstract}}

In kantoren met een grote administratieve workload, zoals een advocatenkantoor, bestaat er een grote kans dat er een heel groot aantal aan documenten 
in het archief en verschillende databanken resideert.  Het is niet altijd makkelijk voor advocaten en medewerkers om zich aan te passen aan het snel evoluerende klimaat van informatietechnologie omdat zij zich vooral 
focussen op het (ook snel evoluerende) rechtssysteem. Zodoende stagneert (of evolueert) de kwaliteit van de ondersteuning, maar de productiviteit daarentegen kan verlagen.  \\

Daarom kan het handig zijn voor dergelijke kantoren om te investeren in een systeem dat dienst kan doen als een digitale assistent. 
Deze bachelorproef gaat over de implementatie van dergelijk systeem. 
Hij is verdeeld in in verschillende stappen. 
Deze stappen zullen de rode draad vormen in deze proef en gaan over een analyse van het kantoor en mogelijke bottlenecks, het vergaren van trainingdata, het bouwen van een server en interface
voor advocaten om te werken met een tool die uiteindelijk hun productiviteit zal verhogen. \\  

Ook zal er analyse gedaan worden naar welke technologieën er voor handen zijn en welke er het best passen bij de toepassingen. 
Bijvoorbeeld zal er onderzocht worden welk type database het snelst kan omgaan met de data die gebruikt wordt, 
welke programma's er passend zijn voor implementatie, ... 
Ook wordt via de eerste analyse bekeken welke toepassingen haalbaar zijn in het tijdskader van 12 weken. \\

Achteraf volgt een korte analyse, die stukken van het Business Acceptance Model gebruikt om te schetsen wat de impact is.
