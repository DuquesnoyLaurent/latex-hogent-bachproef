\chapter{Context-aware chunking van verzamelde data}
Nu alle data in een juist formaat en met metadata in de database steekt, is het tijd voor het volgende ordeningsprincipe: chunking.
Chunking is het principe van extensieve ongestructureerde data op te delen in kleinere, logische stukken, oftewel chunks.
Dit proces kan verdeeld worden in verschillende stappen.

\begin{enumerate}
	\item \textbf{Voorbereiden en reinigen van data:} Bekijk alle data en verwijder onnodige opmaak, titels en punctuatie. Deze stap is optioneel maar het
	      kan altijd helpen om de verzamelde data eens te overlopen en zien of er geen grote gaten in zitten.
	\item \textbf{Tokenization van de tekst:} De data kan nu onderverdeeld worden in tokens, oftewel individuele woorden.
	      Dit kan bereikt worden met algoritmes zoals SpaCy of NLTK,
	      dit zijn Natural Language Processing algoritmes.
	\item \textbf{Het detecteren van zinnen:} Nu de data verdeeld is in tokens, is het tijd om met de hulp van sentence boundary algoritmes de zinnen te
	      detecteren die coherente zinnen vormen. Er zijn ook veel NLP bibliotheken die deze algoritmes implementeren.
	\item \textbf{Groeperen van zinnen in chunks:} Nu de zinnen verdeeld zijn, is het tijd om deze te verdelen in groepen, oftewel chunks.
	      De manier waarop dit kan gebeuren kan via een woordlimiet per chunk, een specifieke topic per chunk of een semantische/ syntactische gelijkenis van zinnen.
\end{enumerate}

Nu deze chunks gegroepeerd zijn, vervangen we de data door de chunks in de database en slaan we ze op samen met de metadata van de documenten.
Deze chunks kunnen voorgesteld worden als vectoren en kunnen samen geïmplementeerd worden in een vector database.
Dit type database slaat vectoren op op basis van hun similariteit met andere vectors en deze databases gebruiken veelal het Approximate Nearest Neighbour algorithm.

Een citaat van rockset.com verduidelijkt dit principe:

\begin{displayquote}
	With the evolution of machine learning, neural networks and large language models, organizations can easily transform unstructured data into embeddings, commonly represented as vectors.
	Vector search operates across these vectors to identify patterns and quantify similarities between components of the underlying unstructured data. 
	\autocite{Rockset}
\end{displayquote}

Nu de data verdeeld is in chunks, gebruiken we een embeddings model om van iedere chunk een vectorrepresentatie te maken en deze slaan we op in een vector store. 

Dergelijke embeddings modellen kunnen gevonden worden op HuggingFace. Vanaf hier kunnen we libraries zoals LangChain om alles te wrappen. 

Nu kunnen we verdergaan naar de volgende stap, het integreren van Ollama met onze database.
