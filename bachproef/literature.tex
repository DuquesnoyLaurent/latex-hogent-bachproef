\chapter{Literatuurstudie}
De gevoerde literatuurstudie in hoofdstuk \ref{ch:stand-van-zaken} is een heel elementair deel van een bachelorproef voor verschillende redenen:

\begin{itemize}
	\item \textbf{Het informeren van de lezer zodat hij comfortabel deze bachelorproef kan doornemen met de vers verworven technische kennis}
	\item \textbf{Het schetsen van context rond ons onderzoeksonderwerp}
	\item \textbf{Het afleiden en identificeren van problemen uit de onderzoeksvraag}
	\item \textbf{Het onderzoeken van mogelijke oplossingen op deze problemen}
\end{itemize}

Een literatuurstudie is ook een heel belangrijk onderwerp om te checken of het concept en de problemen matchen met real-life situaties. 
Het opzoeken van relevante informatie online staat ons ook toe om het onderzoek over efficiëntere boegen te gooien en de juiste richting in te sturen.\\ 

Na de literatuurstudie hebben we deze taken voltooid en hebben we een goeie fundering gebouwd om te beginnen aan een vergelijkende studie naar de beste implementatiestrategieën voor een digitale assistent. 
