%%=============================================================================
%% Conclusie
%%=============================================================================

\chapter{Conclusie}%
\label{ch:conclusie}

% TODO: Trek een duidelijke conclusie, in de vorm van een antwoord op de
% onderzoeksvra(a)g(en). Wat was jouw bijdrage aan het onderzoeksdomein en
% hoe biedt dit meerwaarde aan het vakgebied/doelgroep? 
% Reflecteer kritisch over het resultaat. In Engelse teksten wordt deze sectie
% ``Discussion'' genoemd. Had je deze uitkomst verwacht? Zijn er zaken die nog
% niet duidelijk zijn?
% Heeft het onderzoek geleid tot nieuwe vragen die uitnodigen tot verder 
%onderzoek?

De uitkomst van dit onderzoek is een roadmap die informatie bevat over iedere component van de opbouw van een digitale assistent. 
Eerst was de opzet van dit onderzoek om een assistent te implementeren en te integreren met bestaande software. 

Natuurlijk werkt deze assistent met libraries en componenten die toch heel wat resources gebruiken om vloeiend te bouwen, draaien en testen. 
Dit in combinatie met de gevoeligheid van de cliëntdata is de beslissing gekomen om dit onderzoek puur hypothetisch uit te voeren. 
Denk over dit onderzoek als een stappenplan om een lezer bekend te maken met de basisconcepten van Natural Language Processing, Large Language Models, Retrieval Augmented Generation en meer. 

Veel stappen uit dit onderzoek worden minder breed besproken gezien het hedendaagse aanbod van digitale tools zoals JavaScript libraries, 
frameworks die al heel goeie implementaties voor digitale assistenten leveren en de extraordinaire snelle aard van verandering van het huidig technologisch spectrum. 

Er is in dit onderzoek geprobeerd om zo veel mogelijk technologiespecifiek te werken. 
Het kan goed zijn dat er binnenkort betere technologieën uitgebracht worden die veel beter presteren dan degene die hier gebruikt worden. 
De technologieën gebruikt in dit onderzoek zijn op het moment van onderzoek de meest geschikte en performante op de markt, waarvan de meeste open-source. 

In conclusie hoop ik dat een lezer van dit onderzoek enerzijds bijleert over digitale assistenten en hoe ze werken onder de motorkap. 
Anderzijds dat hij het ook kan gebruiken als stappenplan tijdens het implementeren van zijn eigen assistent, iets waar de student die er nu aan werkt ook zeker gebruik van zal maken. 
