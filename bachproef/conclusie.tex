%%=============================================================================
%% Conclusie
%%=============================================================================

\chapter{Conclusie}%
\label{ch:conclusie}

% TODO: Trek een duidelijke conclusie, in de vorm van een antwoord op de
% onderzoeksvra(a)g(en). Wat was jouw bijdrage aan het onderzoeksdomein en
% hoe biedt dit meerwaarde aan het vakgebied/doelgroep? 
% Reflecteer kritisch over het resultaat. In Engelse teksten wordt deze sectie
% ``Discussion'' genoemd. Had je deze uitkomst verwacht? Zijn er zaken die nog
% niet duidelijk zijn?
% Heeft het onderzoek geleid tot nieuwe vragen die uitnodigen tot verder 
%onderzoek?

De uitkomst van deze bachelorproef is een proof-of-concept van een chatbot die gebruikt kan worden door een advocatenkantoor om hun workflow te boosten 
door het elimineren van repetitieve taken zoals invorderingen schrijven.\\  

Hoewel deze chatbot nog niet op punt staat, is er zeker nog ruimte om deze verder uit te werken door middel van
het prompt en de brondata aan te passen om betere resultaten te verkrijgen.\\ 

Natuurlijk werkt deze assistent met libraries en componenten die toch heel wat resources gebruiken om vloeiend te bouwen, draaien en testen. 

Bepaalde stappen uit dit onderzoek (zoals de frontend-implementatie) worden minder breed besproken gezien het hedendaagse aanbod van digitale tools zoals JavaScript libraries en frameworks die 
al heel goeie implementaties voor digitale assistenten leveren en de extraordinaire snelle aard van verandering van het huidig technologisch spectrum. \\ 

Er is in dit onderzoek geprobeerd om zo veel mogelijk technologiespecifiek te werken. 
Het kan goed zijn dat er binnenkort betere technologieën uitgebracht worden die veel beter presteren dan degene die hier gebruikt worden. 
De technologieën gebruikt in dit onderzoek zijn op het moment van onderzoek de meest geschikte en performante op de markt, waarvan de meeste open-source. \\

In conclusie hoop ik dat een lezer van dit onderzoek enerzijds bijleert over digitale assistenten en hoe ze werken onder de motorkap en anderzijds 
geïnspireerd wordt om misschien zijn/haar eigen digitale assistent te bouwen.  
