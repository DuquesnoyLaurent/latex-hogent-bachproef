%%=============================================================================
%% Voorwoord
%%=============================================================================

\chapter*{\IfLanguageName{dutch}{Woord vooraf}{Preface}}%
\label{ch:voorwoord}

%% TODO:
%% Het voorwoord is het enige deel van de bachelorproef waar je vanuit je
%% eigen standpunt (``ik-vorm'') mag schrijven. Je kan hier bv. motiveren
%% waarom jij het onderwerp wil bespreken.
%% Vergeet ook niet te bedanken wie je geholpen/gesteund/... heeft

Beste lezer, als u dit leest hebt u een kopie bemachtigd van mijn bachelorthesis. 
Ik zou eerst en vooral mijn promotor, de heer Jan Claes van HoGent en mijn co-promotor, 
Stijn Dewolf van Deltalex van harte willen bedanken voor de goede ondersteuning en behulpzaamheid tijdens dit project. 

Dit is een thesis geschreven door een student die een passie heeft in het gebied van Informatietechnologie en er al heel zijn leven mee bezig is. 
Deze passie heeft mij toegelaten om te denken als een problem solver en problemen rond mij te zoeken, vast te stellen en vervolgens te proberen er een oplossing voor te bedenken. 
Het probleem dat ik in deze thesis zal bestuderen is het probleem van repetitie op de werkvloer, meer specifiek in een advocatenkantoor genaamd Deltalex. 
Ik merkte op dat er daar heel veel taken gebeuren die heel repetitief zijn in aard. 

Als programmeur heb ik geleerd dat automatisatie een krachtig denkpatroon is, dus dacht ik dan ook om dit probleem eens onder de loep te nemen. 

Mogelijke oplossingen die mij te binnen schoten waren plugins in document editors (b.v. Word), custom document macro's e.d. 

Na wat feedback van mijn promotor ben ik op het idee gekomen om een digitale assistent te bouwen om de advocaten in allerlei administratieve taken bij te staan. 

Nu is het probleem dat er bij advocaten heel veel gewerkt wordt met confidentiële data. 
Het is dus imperatief dat de data van cliënten nooit de veilige omgeving van Deltalex verlaat. 
Om deze regel te respecteren heb ik een digitale assistent gebouwd die volledig lokaal op een server kan draaien. 
De bedoeling van deze thesis is dus om u, de lezer, zo goed mogelijk te informeren over mijn afgelegde route in het ontwikkelen van deze assistent. 
Ik hoop dan ook om u zo veel mogelijk bij te kunnen leren tijdens dit avontuur van automatisatie en de verkenning van Artificiële intelligentie. 
