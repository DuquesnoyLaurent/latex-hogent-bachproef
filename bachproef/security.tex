\chapter{Haalbaarheid en dataveiligheid}
\label{ch:feasability}
Hoe haalbaar is een dergelijke chatbotimplementatie? 
Tot hoeverre kan een chat app geïmplementeerd worden totdat er kostbare cliëntinformatie op het spel komt te staan? 
Omdat deze bachelorproef geschreven is door een student met vrijwel nul ervaring in dataveiligheid kan er op geen enkele manier geëxperimenteerd worden met confidentiële clientdata. \\ 

Daarom blijft dit onderzoek hypothetisch en overlopen we de elementaire stappen in het bouwen van een implementatie die een advocaat zal toestaan om bepaalde repetitieve taken, 
zoals opzoeken van data in een gesloten databank en opstellen van teksten, te gaan automatiseren en zodoende te versnellen in uitvoering om tijd uit te sparen voor belangrijk opzoekingswerk, 
rechtbankwerk, cliëntconsultaties e.d.\\ 

Een van de vooraanstaande redenen dat er geen reële implementatie komt is omdat Large Language RAG-modellen heel zwaar zijn voor een computer om uit te voeren. 
Er zou bij een computer al een heel zware grafische kaart moeten aanwezig zijn om verschillende connecties van advocaten tegelijkertijd te kunnen valideren. 
Als men dit in een cloudgebaseerde applicatie zou draaien, zou dit een mogelijk risico kunnen vormen om een clouddienst toegang te geven tot cliëntdata.
Ook kan iemand met slechte intenties netwerkverkeer onderscheppen en zodoende (confidentiële) data bemachtigen. \\ 

Om een lokale oplossing te draaien is het best om een heel sterke grafische kaart te installeren in de server op het lokale netwerk. 
Prijzen van (capabele) NVIDIA GPU's (GeForce RTX 3080 ≤) starten in België (tijdens het schrijven van deze paper) rond de 550 EUR. 
Het zou dus een preliminaire investering vereisen van een kantoor om absolute dataveiligheid te garanderen. 
