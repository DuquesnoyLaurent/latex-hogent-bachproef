\chapter{Testing}

Testing van onze RAG chat tool is een proces dat oppervlakkig verdeeld kan worden in drie hoofdstukken. 

\section{Testen van het systeem}
Een eerste stap is \textbf{functioneel testen}, dit is de meest basic vorm van testen en kijkt of alle functies van het systeem al dan niet werken. 
Denk maar aan gebruikersinput, query output, of het retrievalsysteem wel relevante chunks ophaalt, ...
Daarna komt \textbf{performance testing}. Dit evalueert of het systeem onder verschillende loads naar behoren presteert. 
Dit type testen wordt gebruikt om mogelijke bottlenecks te elimineren en een performant systeem te bekomen. 
Als laatst hebben we \textbf{user experience testing}. Hier verwerken we feedback van de gebruikers over de bruikbaarheid van het systeem. 
Dit houdt zaken in zoals lay-out, snelheid, gebruikersvriendelijkheid, ... 

\section{Optimalisatie van het systeem}
Dit hoofdstuk kan eigenlijk oneindig lang herhaald worden. 
Er zijn altijd dingen die betere kunnen en we kunnen ze identificeren door volgende technieken toe te passen:

\begin{itemize}
	\item \textbf{Iteratieve optimalisatie:} Het constant verwerken van de feedback van advocaten en observeren van de evolutie van data kan de antwoorden van onze assistent accurater en behulpzamer maken. 
	\item \textbf{Veranderen van het model:} Als je een beetje de actualiteit rond LLMs volgt, kan het zijn dat er plots eentje op de proppen komt die llama3 zal outperformen. 
		Het is aangeraden om hierover up-to-date te blijven. 
	\item \textbf{Robuustheid:} Observeren van logbestanden van Langchain kan ons naar memory leaks, fouten in chunking of embeddings en andere leiden. 
		Robuustheid kan ook verbeterd worden door geoptimaliseerde foutbehandeling. 
\end{itemize}

Deze lijst kan ook toegepast worden als we willen schalen naar grotere datasets/ een groter aantal gebruikers. 

\section{Monitoring en feedback}
Zoals in het vorige hoofdstuk al aangewezen, is het van elementair belang dat we het systeem constant in de gaten houden. 
Dit kan door rechtstreeks logs uit te lezen, of deze te sturen naar een logging- en visualisatieframework zoals Grafana.
Data is alles en op basis van die data kunnen we blijven optimaliseren en verbeteren. 
