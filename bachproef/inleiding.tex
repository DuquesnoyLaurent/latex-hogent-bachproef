%%=============================================================================
%% Inleiding
%%=============================================================================
\chapter{\IfLanguageName{dutch}{Inleiding}{Introduction}}

\section{\IfLanguageName{dutch}{Probleemstelling}{Problem Statement}}%
\label{sec:probleemstelling}

Als we kijken bij de administratieve backend in Deltalex advocaten en observeren hoe bepaalde taken gebeuren, valt op dat veel taken die uitgevoerd worden repetitief en tijdrovend zijn.
Deze taken kunnen impact hebben op de productiviteit van medewerkers, de herhalende aard van de taken kan de kans op fouten die makkelijk overzien worden vergroten. \\

Deze kunnen een grote impact hebben op het kantoor.
Denk maar aan foute looncalculaties, miscommunicatie, spelfouten, \dots


\section{\IfLanguageName{dutch}{Onderzoeksvraag}{Research question}}%
\label{sec:onderzoeksvraag}

Uit de bovenstaande passage ontspringt de vraag:
"Op welke manieren kunnen repetitieve taken geautomatiseerd worden?
Welke technologieën zijn hiervoor passend en hoe kunnen ze worden geïmplementeerd?
In welke mate kan dergelijke toepassing de productiviteit van een advocaat positief beïnvloeden?"

\newpage

\section{\IfLanguageName{dutch}{Onderzoeksdoelstelling}{Research objective}}%
\label{sec:onderzoeksdoelstelling}
Het beoogde resultaat van deze bachelorproef kan opgedeeld worden in een paar delen:
\begin{itemize}
	\item \textbf{Onderzoeken wat de bottlenecks zijn in een huidig kantoorklimaat}
	\item \textbf{Onderzoeken wat de technologieën zijn die ons hierbij kunnen helpen}
	\item \textbf{Onderzoeken hoe deze kunnen samen gegoten worden in een toepassing die mikt op het verbeteren van de productiviteit van een advocaat door repetitieve taken te automatiseren}
	\item \textbf{Bekijken wat de impact is van een demoversie bij een advocaat}
\end{itemize}

De onderzoeksvraag zal ontleed worden in deelproblemen en de bedoeling van deze bachelorproef is
om te bewijzen dat er technologieën bestaan die ons van dienst kunnen zijn om deze hardnekkige problemen uit de weg te ruimen.

\section{\IfLanguageName{dutch}{Indeling van deze bachelorproef}{Structure of this bachelor thesis}}%
\label{sec:opzet-bachelorproef}

% Het is gebruikelijk aan het einde van de inleiding een overzicht te
% geven van de opbouw van de rest van de tekst. Deze sectie bevat al een aanzet
% die je kan aanvullen/aanpassen in functie van je eigen tekst.

De structuur van deze bachelorproef is zoals volgt opgebouwd:\\

In hoofdstuk \ref{ch:stand-van-zaken} wordt op basis van een literatuurstudie context gegeven over het probleem door het in verschillende secties op te delen.
Per sectie van het probleem wordt een mogelijke oplossing onderzocht en ondersteund aan de hand van gevonden relevante literatuur. \\

In het hoofdstuk methodologie(\ref{ch:methodologie}) wordt besproken hoe deze bachelorproef er uit zal zien en hoe er te werk wordt gegaan.  \\

Daarna volgt hoofdstuk \ref{ch:technologies}, waar de technologieën gebruikt in de implementatie besproken worden. Ook wordt de keuze ervoor gemotiveerd. \\

In hoofdstuk \ref{ch:implementation} wordt de implementatie van de proof-of-concept stap voor stap overlopen en beschreven.\\ 

In hoofdstuk \ref{ch:results_reflection} worden resultaten van de implementatie besproken.
Alsook wordt de assistent beoordeeld door een advocaat van Deltalex en geeft hij zijn mening of onze implementatie wel degelijk haar doel heeft bereikt. \\

Tenslotte wordt in hoofdstuk \ref{ch:conclusie} gepraat over hoe het onderzoek ging en of er bereikt is wat er initieel verwacht werd.
