%%=============================================================================
%% Inleiding
%%=============================================================================

\chapter{\IfLanguageName{dutch}{Inleiding}{Introduction}}%
Deze bachelorproef handelt over het verbeteren van de efficiëntie van administratieve taken bij advocatenkantoren.
De natuur van de taken die deze beoogt te verbeteren zijn veelal repetitief in aard.
Dit onderzoek ontspringt zich in de observatie dat bepaalde taken die uitgevoerd worden frustrerend en traag zijn.
Een digitale assistent kan de tijd die deze taken innemen minimaliseren en het mogelijk maken deze te herinvesteren in taken die meer uitdagend en variërend zijn.

\section{\IfLanguageName{dutch}{Probleemstelling}{Problem Statement}}%
\label{sec:probleemstelling}

Als we kijken bij de administratieve backend in Deltalex advocaten en observeren hoe bepaalde taken gebeuren, valt op dat veel taken die uitgevoerd worden repetitief en tijdrovend zijn.
Deze taken kunnen impact hebben op de productiviteit van medewerkers, de herhalende aard van de taken kan de kans op fouten die makkelijk overzien worden vergroten.
Deze kunnen een grote impact kunnen hebben op het kantoor.
Denk maar aan foute looncalculaties, miscommunicatie, spelfouten, \dots
Natuurlijk is het niet eenvoudig om een automatisatie toe te passen op een heel specifiek punt, deze bachelorproef zal deels dienen als een Proof Of Concept om de haalbaarheid van
dergelijke tools in een geavanceerde kantooromgeving te illustreren.

\section{\IfLanguageName{dutch}{Onderzoeksvraag}{Research question}}%
\label{sec:onderzoeksvraag}

Uit de bovenstaande passage ontspringt de vraag:
"Op welke manieren kunnen repetitieve taken geautomatiseerd worden?
Welke technologieën zijn hiervoor passend en hoe kunnen ze worden geïmplementeerd?
In welke mate kan dergelijke toepassing de productiviteit van een advocaat positief beïnvloeden?"

\section{\IfLanguageName{dutch}{Onderzoeksdoelstelling}{Research objective}}%
\label{sec:onderzoeksdoelstelling}
Het beoogde resultaat van deze bachelorproef kan opgedeeld worden in een paar delen:
\begin{itemize}
	\item \textbf{Onderzoeken wat de bottlenecks zijn in een huidig kantoorklimaat}
	\item \textbf{Onderzoeken wat de technologieën zijn die ons hierbij kunnen helpen}
	\item \textbf{Onderzoeken hoe deze kunnen samen gegoten worden in een toepassing die mikt op het verbeteren van de productiviteit van een advocaat door repetitieve taken te automatiseren}
\end{itemize}


\section{\IfLanguageName{dutch}{Opzet van deze bachelorproef}{Structure of this bachelor thesis}}%
\label{sec:opzet-bachelorproef}

% Het is gebruikelijk aan het einde van de inleiding een overzicht te
% geven van de opbouw van de rest van de tekst. Deze sectie bevat al een aanzet
% die je kan aanvullen/aanpassen in functie van je eigen tekst.

De structuur van deze bachelorproef is zoals volgt opgebouwd:

In Hoofdstuk~\ref{ch:stand-van-zaken} wordt op basis van een literatuurstudie context gegeven over het probleem door het in verschillende secties op te delen. 
Per sectie van het probleem wordt een mogelijke oplossing onderzocht en ondersteund aan de hand van gevonden relevante literatuur. 

In Hoofdstuk~\ref{ch:methodologie} volgt een opsomming van de stappen die ondernomen kunnen worden om een digitale AI-assistent te bouwen op basis van een grote documentendatabase. 
De volgende hoofdstukken zijn uiteenzettingen van de stappen. 
Deze gaan dieper in op het onderwerp en schetsen stap per stap hoe ze verwezenlijkt kunnen worden. 
Met andere woorden wordt er stap per stap uitgelegd hoe men een digitale assistent kan bouwen op basis van de technologieën gevonden in hoofdstuk \ref{ch:stand-van-zaken}.

Tenslotte wordt in hoofdstuk~\ref{ch:conclusie} een conclusie gegeven en teruggeblikt op het onderzoek. 
Daarbij wordt ook een aanzet gegeven voor toekomstig onderzoek binnen dit domein.
