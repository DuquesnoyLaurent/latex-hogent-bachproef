%%=============================================================================
%% Inleiding
%%=============================================================================

\chapter{\IfLanguageName{dutch}{Inleiding}{Introduction}}%
Deze bachelorproef handelt over het verbeteren van de efficiëntie van administratieve taken bij advocatenkantoren.
De natuur van de taken die deze beoogt te verbeteren zijn veelal repetitief in aard.
Dit onderzoek ontspringt zich in de observatie dat bepaalde taken die uitgevoerd worden frustrerend en traag zijn.
Een technologisch onderzoek heeft een hoge kans om de tijd die deze taken innemen vrij te maken en mogelijk te herinvesteren in taken die meer uitdagend zijn en variëren in aard.

\section{\IfLanguageName{dutch}{Probleemstelling}{Problem Statement}}%
\label{sec:probleemstelling}

Als we kijken bij de administratieve backend in Deltalex advocaten en observeren hoe bepaalde taken gebeuren, valt op dat veel taken die uitgevoerd worden repetitief en tijdrovend zijn.
Deze taken kunnen impact hebben op de productiviteit van medewerkers, de herhalende aard van de taken vergroot
ook de kans op fouten die makkelijk overzien worden maar een grote impact kunnen hebben op het kantoor.
Denk maar aan foute looncalculaties, miscommunicatie, spelfouten, \dots
Natuurlijk is het niet eenvoudig om een automatisatie toe te passen op een heel specifiek punt, deze bachelorproef zal deels dienen als een Proof Of Concept om de haalbaarheid van
dergelijke tools in een geavanceerde kantooromgeving te illustreren.

\section{\IfLanguageName{dutch}{Onderzoeksvraag}{Research question}}%
\label{sec:onderzoeksvraag}

Uit de bovenstaande passage ontspringt de vraag:
"Bestaat er een mogelijkheid om repetitieve taken te automatiseren?
Kan een dergelijke toepassing de productiviteit van een advocaat positief beïnvloeden?
Hoe haalbaar is een dergelijke toepassing?
Kan deze dienen als vervanging van de manuele uitvoer van zulke taken?".

\section{\IfLanguageName{dutch}{Onderzoeksdoelstelling}{Research objective}}%
\label{sec:onderzoeksdoelstelling}
Wat is het beoogde resultaat van je bachelorproef?
Wat zijn de criteria voor succes? Beschrijf die zo concreet mogelijk. Gaat het bv. om een proof-of-concept, een prototype, een verslag met aanbevelingen, een vergelijkende studie, enz.

Het beoogde resultaat van deze bachelorproef kan opgedeeld worden in zes delen:
\begin{itemize}
	\item Het versnellen en verbeteren van de workflow van advocaten die werkzaam zijn bij Deltalex
    \item Het analyseren van de data (dossiers, communicatie, documenten) die gebruikt kan worden voor optimalisatie
    \item Naast een digitale assistent, een potentiële snelle, geoptimaliseerde zoekmachine voor hun documentendatabase
    \item De haalbaarheid van deze features in kaart brengen
    \item Aantonen via onderzoek of een dergelijke implementatie concrete voordelen heeft op de manuele executie van taken
\end{itemize}


\section{\IfLanguageName{dutch}{Opzet van deze bachelorproef}{Structure of this bachelor thesis}}%
\label{sec:opzet-bachelorproef}

% Het is gebruikelijk aan het einde van de inleiding een overzicht te
% geven van de opbouw van de rest van de tekst. Deze sectie bevat al een aanzet
% die je kan aanvullen/aanpassen in functie van je eigen tekst.

De rest van deze bachelorproef is als volgt opgebouwd:

In Hoofdstuk~\ref{ch:stand-van-zaken} wordt een overzicht gegeven van de stand van zaken binnen het onderzoeksdomein, op basis van een literatuurstudie.

In Hoofdstuk~\ref{ch:methodologie} wordt de methodologie toegelicht en worden de gebruikte onderzoekstechnieken besproken om een antwoord te kunnen formuleren op de onderzoeksvragen.

% TODO: Vul hier aan voor je eigen hoofstukken, één of twee zinnen per hoofdstuk

In Hoofdstuk~\ref{ch:conclusie}, tenslotte, wordt de conclusie gegeven en een antwoord geformuleerd op de onderzoeksvragen. Daarbij wordt ook een aanzet gegeven voor toekomstig onderzoek binnen dit domein.
